\documentclass{book}
\usepackage[utf8]{inputenc}
\usepackage[T1]{fontenc}
\usepackage{lmodern}
\usepackage[a4paper,left=3.5cm,right=2.5cm,top=2.5cm,bottom=2cm]{geometry}
\usepackage[frenchb]{babel}
\usepackage{fncychap}
\usepackage{fancyhdr}
\usepackage{sectsty}
\usepackage[pdftitle={Master complémentaire en droit fiscal - Droit pénal des entreprises},pdfauthor={Laurent RICHARD}, pdfsubject={Notes de cours}, pdfkeywords={ULg, Fisc}]{hyperref}
\usepackage{graphicx}
\usepackage{pdfpages}
\usepackage{listings}
\usepackage{lscape}
\allsectionsfont{\sffamily}
\usepackage{fancybox}
\setcounter{secnumdepth}{4}
\usepackage{latexsym}
\usepackage{pifont}
\renewcommand\FrenchLabelItem{\textbullet}
\widowpenalty=9999
\clubpenalty=9999




\begin{document}
\sffamily
\newcommand{\RPoint}{\protect\includegraphics[height=1.7ex,keepaspectratio]{point.png}}
\newcommand{\RSave}{\protect\includegraphics[height=1.7ex,keepaspectratio]{Save.png}}
\renewcommand\labelitemii{\ding{220}}
\begin{titlepage}

\begin{center}
\begin{Large}ULg - Master complémentaire en droit fiscal\end{Large}\\
\vspace{0.5cm}- \\
\vspace{0.5cm}
Fiscalité des entreprises

\end{center}
\vspace{6cm}

\LARGE
\begin{center}
\textsc{Notes de cours - Droit pénal des entreprises}\\
\end{center}

\vspace{10.0cm}

\normalsize
\flushright{
\textbf{Laurent RICHARD}\\
Comptable-fiscaliste agréé IPCF\\
\vspace{0.5cm}
Année académique 2012-2013}

\end{titlepage}

\newpage

\thispagestyle{empty}
\setcounter{page}{0}
\null

\newpage
\thispagestyle{empty}
\setcounter{page}{0}
\vspace{20cm}

\vfill
\begin{flushright}
Life is a succession of lessons, \\
which must be lived to be understood. \\ 
--- \textit{Ralph Waldo Emerson}
\end{flushright}
\vfill
\newpage

\renewcommand{\chaptermark}[1]{\markboth{#1}{}}
\renewcommand{\sectionmark}[1]{\markright{\thesection\ #1}}
\fancyhf{} \fancyhead[LE,RO]{\bfseries\thepage}
\fancyhead[LO]{\bfseries\rightmark}
\fancyhead[RE]{\bfseries\leftmark}
\renewcommand{\headrulewidth}{0.5pt}
\addtolength{\headheight}{0.5pt}
\renewcommand{\footrulewidth}{0pt}
\fancypagestyle{plain}{ \fancyhead{}
\renewcommand{\headrulewidth}{0pt}}

\thispagestyle{empty}
\setcounter{page}{0}
\null
\newpage
\pagenumbering{roman} \setcounter{page}{1} 




\tableofcontents



\chapter{Cours du 14.11.2012}
\pagenumbering{arabic} \setcounter{page}{1} 

\section{Références}
Le droit pénal des affaires, Spreutel, Roggen & Roger Frank, DPA, Bruylant, 2005

Guide juridique de l'entreprise.

Revue : Droit pénal de l'entreprise. Larcier. Trimestriel
\section{Introduction}

Les acteurs de la vie des affaires\\
La morale générale ou le civisme général\\
La déontologie concernant les professions organisées.\\
Les règles pénales. C'est un tribunal répressif qui va juger sur base d'un texte de loi.\\

Il y a aussi les règles de bonnes pratiques. Code des bons usages.		\\

Le DPE s'applique aux personnes physiques et aux personnes morales. Il s'intéresse à la vie de l'entreprise (faux en écriture, ...) mais aussi à sa création (législation sur la BCE, accès à la profession) et sa mort (ne pas avoir fait aveu de faillite, détournement d'actif), PRJ.\\

L'infraction la plus courante est le faux en écriture, infraction de blanchiment, escroquerie, bénéfice au détriment de l'environnement. On est dans une criminalité de profit. La criminalité de profit appelle une réponse adéquate -> Confiscation et amendes financières. Il faut donc un arsenal juridique pour aller chercher l'argent où il est.\\

Il existe aussi le droit pénal des entreprises publiques (intercom, entreprises publiques autonomes, ...). Dont le délit d'ingérance (C.Pénal - article 245). Prise d'intérêt.\\

Pas mal d'infractions hors du C.Penal, loi en matière commerciale, sociale, fiscale qui comportent aussi des infraction (contre-façon en matière de droit d'auteur). Contrefaçon, droit pénal boursier, délit d'initié, loi Prospectus, distribution de dividendes fictifs. Faux dans les rapports de gestion, A.G., Droit pénal des transport, temps de conduite, règlement EU des transports. Droit pénal fiscal, fraude fiscal. Droit pénal de l'environnement, Naufrage de l'Erika. Lois sur les pratiques du commerce.(solde,...) Santé du consommateur (Amiante, prothèse mammaire, sang contaminé, ...) Loi sur le crédit à la consommation. Loi Breyne. Droit pénal social (règlement sur la protection du travail)\\


Le droit pénal a un rôle répressif, sanctionner ceux qui ne respectent pas les règles mais vient en dernier recours. Comportant les plus scandaleux, les moins tolérables. Rôle de protection des victimes (Auditorat du Travail). ex: négriers de la construction, carrousel TVA, escroqueries du type Madoff, corruption. Il y a aussi un rôle préventif, rappeler la norme. Ce qui fait peur c'est la sévérité de la peine et la certitude d'être pris.\\

Outils particuliers : police spécialisée, parquet spécialisée. juge d'instruction spécialisée disposant de pouvoirs particuliers. Article 29 du C.I.C. obligent les fonctionnaires à dénoncer. Effectivité de la peine - emprisonnement, confiscation et peine d'amende. Associé aux saisies, préventives, ....\\

Autres mécanisme pour parvenir aux buts : autres mécanismes de sanctions (sanctions civiles, sanctions administratives, action en cessation), régime préventif en matière de blanchiment, transaction pénale, auto-régulation encadrée.\\

\section{Suite droit pénal}

Tout part d'un fait infractionnel. \\

On va devoir le découvrir via :

\begin{itemize}
\item une dénonciation (anonyme ou non) qui est la porte ouverte pour entamer une enquête.
\item une plainte d'une victime qui a subi un dommage.
\item le travail policier. (regarder les sociétés qui n'ont pas déposé de comptes annuels depuis plus de 3 ans.)
\end{itemize}

Le dossier va être transmis au Parquet qui est un service pour rassembler les preuves et entamer l'action publique. Le ministère public va faire un travail d'information, recherche des auteurs par le substitut du procureur du Roi pour amener le dossier au Tribunal via une citation directe (Tribunal de Police, Tribunal correctionnel, Cour d'Assises). Le substitut va diriger l'enquête, adresser des documents aux services de police. (Police fédérale, PJ, zone de police, ...). Ils font des devoirs d'auditions, entendre des personnes (les victimes, les témoins, de fonctionnaires, de suspects). Il peut aussi faires des perquisitions avec accord de la personne, des saisies, visiter des bâtiments public, prendre des échantillons, désigner un expert afin de  retrouver les auteurs. Il va citer des prévenus à comparaître devant le Tribunal. 2/3 des fois, c'est classé sans suite souvent car auteur inconnu, faits prescrits, les choses sont rentrées dans l'ordre, il y a eu indemnisation. Entre les 2, il peut y avoir une médiation pénale ou une transaction pénale (versement d'une somme d'argent pour éteindre les poursuites).\\

Dans d'autres cas, le juge d'instruction va instruire le dossier et mettre le dossier à l'instruction (moins de 10\% des dossiers). Il va recherches les auteurs et les preuves des faits qui lui sont dénoncés. Il va s'appuyer sur les policiers fédéraux et des zones de police. Il a plus de pouvoirs que le substitut dont des perquisitions sans/contre le consentement des personnes, écoute téléphonique, prélèvement ADN même forcée mais aussi la détention préventive/mandat d'arrêt. Si la détention dure plus de 5 jours, c'est la chambre du Conseil qui va confirmer la détention. A la fin de l'instruction à charge et à décharge, il envoie le dossier au procureur du roi. C'est là qu'on agit via un réquisitoire (une demande) à destination de la Chambre du Conseil(Chambre des mises en Accusation pour le degré d'Appel). Il va libeller les infraction reprochées. La chambre du conseil va pouvoir renvoyer devant le tribunal correctionnel  (ordonnance de renvoi ou ordonnance de non-lieu). Instruction/ Information est secrète (sauf exception). Le travail d'enquete est écrit (P.V., ...) et de manière inquisitoire. Au niveau de l'audience au tribunal, elle est publique et orale. Le suspect est présumé innocent jusqu'à ce qu'il n'ai plus d'appel/recours possible.

Pour toute faillite, le curateur va faire un rapport financier.















\nocite{*}
\bibliographystyle{plain}
\bibliography{publications2}       % 'publications' is the name of a BibTeX file
\addcontentsline{toc}{chapter}{Bibliographie} 
\end{document}