\documentclass{book}
\usepackage[utf8]{inputenc}
\usepackage[T1]{fontenc}
\usepackage{lmodern}
\usepackage[a4paper,left=3.5cm,right=2.5cm,top=2.5cm,bottom=2cm]{geometry}
\usepackage[frenchb]{babel}
\usepackage{fncychap}
\usepackage{fancyhdr}
\usepackage{sectsty}
\usepackage[pdftitle={Master complémentaire en droit fiscal - Principes généraux du Droit fiscal},pdfauthor={Laurent RICHARD}, pdfsubject={Notes de cours}, pdfkeywords={ULg, Fisc}]{hyperref}
\usepackage{graphicx}
\usepackage{pdfpages}
\usepackage{listings}
\usepackage{lscape}
\allsectionsfont{\sffamily}
\usepackage{fancybox}
\setcounter{secnumdepth}{4}
\usepackage{latexsym}
\usepackage{pifont}
\renewcommand\FrenchLabelItem{\textbullet}
\widowpenalty=9999
\clubpenalty=9999




\begin{document}
\sffamily
\newcommand{\RPoint}{\protect\includegraphics[height=1.7ex,keepaspectratio]{point.png}}
\newcommand{\RSave}{\protect\includegraphics[height=1.7ex,keepaspectratio]{Save.png}}
\renewcommand\labelitemii{\ding{220}}
\begin{titlepage}

\begin{center}
\begin{Large}ULg - Master complémentaire en droit fiscal\end{Large}\\
\vspace{0.5cm}- \\
\vspace{0.5cm}
Fiscalité des entreprises

\end{center}
\vspace{6cm}

\LARGE
\begin{center}
\textsc{Notes de cours - TVA}\\
\end{center}

\vspace{10.0cm}

\normalsize
\flushright{
\textbf{Laurent RICHARD}\\
Comptable-fiscaliste agréé IPCF\\
\vspace{0.5cm}
Année académique 2012-2013}

\end{titlepage}

\newpage

\thispagestyle{empty}
\setcounter{page}{0}
\null

\newpage
\thispagestyle{empty}
\setcounter{page}{0}
\vspace{20cm}

\vfill
\begin{flushright}
Life is a succession of lessons, \\
which must be lived to be understood. \\ 
--- \textit{Ralph Waldo Emerson}
\end{flushright}
\vfill
\newpage

\renewcommand{\chaptermark}[1]{\markboth{#1}{}}
\renewcommand{\sectionmark}[1]{\markright{\thesection\ #1}}
\fancyhf{} \fancyhead[LE,RO]{\bfseries\thepage}
\fancyhead[LO]{\bfseries\rightmark}
\fancyhead[RE]{\bfseries\leftmark}
\renewcommand{\headrulewidth}{0.5pt}
\addtolength{\headheight}{0.5pt}
\renewcommand{\footrulewidth}{0pt}
\fancypagestyle{plain}{ \fancyhead{}
\renewcommand{\headrulewidth}{0pt}}

\thispagestyle{empty}
\setcounter{page}{0}
\null
\newpage
\pagenumbering{roman} \setcounter{page}{1} 




\tableofcontents



\chapter{Introduction}
\pagenumbering{arabic} \setcounter{page}{1} 


T.V.A. : taxe sur la valeur ajoutée.\\

Institution fiscale incontournable depuis près de quatre décennies, la taxe sur la valeur ajoutée
est un impôt indirect sur le chiffre d'affaires, frappant la consommation des biens et des
services.\\

De création française, la première T.V.A. devient réalité légale dans la loi française du 10
avril 1954, entrée en vigueur le 1er juillet de la même année.\\

Déjà dans cette loi, les deux principales caractéristiques de la T.V.A., qui feront son succès à
travers le monde, sont présentes : la neutralité et le fractionnement de son paiement.\\

Ces vertus, développées ci-après, ainsi que son «~\textit{coût psychologique}~» faible ou indolore pour
le consommateur, ont amené près de cent pays à travers le monde à en adopter les
mécanismes.

\chapter{TAXE SUR LA VALEUR AJOUTÉE : UN IMPÔT DE CONCEPTION COMMUNAUTAIRE}

\section{La naissance de la taxe sur la valeur ajoutée au niveau européen}

Rappelons-le, le Traité de Rome a pour objectif l'établissement d'un marché commun et d'une
union économique et monétaire.\\

En tant que telle, l'harmonisation fiscale entre les Etats membres n'est donc pas un objectif du
traité. Ceci explique d'ailleurs que ses dispositions consacrées à la fiscalité sont relativement
rares.\\

Et pourtant, cette harmonisation fiscale a toujours été considérée comme étant essentielle à la
réalisation du «~marché commun~».\\

Pouvait-on, en effet, dans l'esprit européen, tolérer des entraves (fiscales) nationales à la libre
circulation des marchandises et des services, au sein de la Communauté, entraves consistant
essentiellement à protéger les produits nationaux au détriment des produits des pays voisins ?\\

Ne fallait-il pas trouver un système de fiscalité indirecte dont l'objectif serait d'exclure tous
mouvements de personnes, de marchandises et de services exclusivement motivés par des
considérations fiscales ?\\

Dans ce contexte, définir un impôt indirect harmonisé à l'échelle européenne devenait
incontournable. L’impôt indirect à créer aurait pour objectif de supprimer les entraves
fiscales nationales et, par conséquent, les détournements de trafic et les distorsions de
concurrence.\\

La définition d'un impôt conçu à un échelon supranational n'est pas chose aisée. La fiscalité
est en effet une prérogative nationale à laquelle les Etats tiennent particulièrement, puisqu'elle
touche directement à leurs moyens financiers.

Le point de départ de l'harmonisation fiscale indirecte à l'échelle européenne fut l'adoption
d'un système commun de taxe sur le chiffre d'affaires.\\

Lors de l'entrée en vigueur du Traité de Rome, tous les Etats membres (la Belgique, le Grand-
Duché de Luxembourg, les Pays-Bas, l'Italie et l'Allemagne), à l'exclusion de la France,
appliquaient sur leur territoire un système de taxes cumulatives en cascade. En Belgique, ce
système était repris sous le vocable «~taxes assimilées au timbre~».\\

\subsection{INCONVÉNIENTS DU SYSTÈME DE TAXES CUMULATIVES EN CASCADES}

Le système des taxes cumulatives en cascade était considéré comme insatisfaisant. C'est
ce qui explique que l'Europe a finalement adopté le système de la taxe sur la valeur ajoutée,
appliqué en France depuis plusieurs années.\\

Le système de taxes cumulatives en cascades consistait à percevoir un impôt à chaque étape
du circuit économique, sans aucune possibilité de déduction de la part des intervenants à ce
circuit.\\

Les inconvénients liés aux taxes cumulatives sont multiples :\\

\begin{itemize}

\item la charge fiscale frappant un bien ou un service est d'autant plus lourde que le
nombre d'agents économiques intervenus pour l'industrialisation ou la
commercialisation du bien ou du service est grand;
\item ces taxes cumulatives favorisent l'intégration des entreprises, entraînant ainsi la
disparition des intermédiaires spécialisés;
\item elles entraînent des difficultés techniques tant à l'importation qu'à l'exportation :
\begin{itemize}
\item à l'importation : l'Etat devait mettre sur pied des barrières douanières
suffisantes pour permettre aux produits nationaux de rester concurrentiels;
\item à l'exportation : l'Etat belge devait accorder des ristournes afin de permettre
aux produits nationaux de rester concurrentiels sur les marchés internationaux.
\end{itemize}
\end{itemize}
\null 
 
Illustrons le système des taxes cumulatives en cascade par un exemple. Imaginons une taxe
cumulative de 10\%. Si cinq agents économiques interviennent de la production à la
consommation, l'impôt cumulatif était calculé de la manière suivante (en supposant l'absence
de valeur ajoutée aux différents stades du circuit) :\\



\begin{tabular}{|c|c|c|c|}
  \hline
  & Taxes dues & Prix de vente H.T. & Prix de vente T.T.C. \\
  \hline
  A & 10 & 100 & 110\\
    \hline
B & 11 & 110 & 121\\
  \hline
C & 12 & 121 & 133\\
  \hline
D & 13 & 133 & 146\\
  \hline
E & 15 & 146 & 161\\
  \hline
Total & 61 & 146 & 161\\

  \hline
\end{tabular}

\null 

Les impôts revenant à l'Etat s'élèvent à 61. Pour améliorer sa rentabilité, l'agent E a
évidemment intérêt à fusionner avec plusieurs agents se situant en amont. Supposons qu'il
fusionne avec B, C et D. Le tableau ci-dessus devient :\\
	
\begin{tabular}{|c|c|c|c|}
  \hline
  & Taxes dues & Prix de vente H.T. & Prix de vente T.T.C. \\
  \hline
  A & 10 & 100 & 110\\
    \hline
BCDE & 11 & 110 & 121\\
  \hline
Total & 21 & 110 & 121\\
  \hline
\end{tabular}

\null 

Dans ce second tableau, l'Etat ne touche que 21. Non pas parce que le prix du bien cédé a été
modifié, mais uniquement parce que le circuit économique est plus court.\\

C'est cette absence de neutralité et de transparence de l'impôt qui a principalement décidé les
autorités européennes à abandonner définitivement l'impôt cumulatif en cascade : il n'était pas
acceptable que des entreprises adaptent leur comportement économique sur la base de
considérations purement fiscales.

\subsection{NAISSANCE DU SYSTÈME COMMUN DE LA TAXE SUR LA VALEUR AJOUTÉE}

Abandonner le régime des taxes cumulatives en cascade était une chose, adopter un régime
appelé à le remplacer en était une autre.\\

Le système de la taxe sur la valeur ajoutée, en vigueur en France depuis 1954, présentait les
apparences d'un système nettement plus adéquat que le régime en vigueur dans les cinq autres
Etats membres. Sa neutralité en était l’atout majeur.\\

C'est ainsi que la T.V.A. vit le jour au niveau européen le 11 avril 1967, lors de l'adoption par
le Conseil des deux premières directives T.V.A.\\

Comme le révèle le préambule de la première directive, l'adoption d'un système commun de
taxe sur la valeur ajoutée poursuivait comme objectif :\\

\begin{itemize}
\item la réalisation d'un marché commun comportant une saine concurrence et ayant des
caractéristiques analogues à celles d'un marché intérieur;
\item l'élimination, dans toute la mesure du possible, des facteurs qui sont susceptibles
de fausser les conditions de concurrence, tant sur la plan national que sur le plan
communautaire;
\end{itemize}

\null

En adoptant un mécanisme unique de taxe sur le chiffre d'affaires, les Etats membres
renonçaient à percevoir toute autre taxe comparable sur leur territoire. L'article 1er de la
première directive prévoyait en ce sens que «~\textit{les Etats membres remplacent leur système
actuel de taxes sur le chiffre d'affaires par le système commun de la taxe sur la valeur ajoutée
(...})~».\\

Cette renonciation était là une garantie du fonctionnement correct du système de la taxe sur la
valeur ajoutée.\\

La Belgique, l'Allemagne, l'Italie, les Pays-Bas et le Luxembourg durent ainsi renoncer à
l'application, sur leur territoire, du système de taxes cumulatives à cascade.\\

\chapter{L'harmonisation progressive de la T.V.A.}

La taxe sur la valeur ajoutée touchant de près à la souveraineté nationale des Etats membres, il
allait de soi que la mise en place de cet impôt indirect devait être progressive. C'est là le prix
du temps à payer pour aboutir à une fiscalité indirecte harmonisée.\\

Au niveau européen, quatre moments importants marquèrent l'évolution de la taxe sur la
valeur ajoutée : sa mise en place par les première et deuxième directives,
l'adoption de la sixième directive , l'adoption du régime transitoire entré en vigueur
le 1er janvier 1993 et, enfin, l’adoption de la directive T.V.A. le 28 novembre 2006.\\

\section{Les première et deuxième directives}

Les fondements de la T.V.A., tels qu'on les connaît aujourd'hui, furent définis par la première
directive.\\

Ils se résument en deux idées :\\

\begin{itemize}
\item le principe du système commun de taxe sur la valeur ajoutée est d'appliquer aux
biens et aux services un impôt général sur la consommation exactement
proportionnel au prix des biens et des services, quel que soit le nombre des
transactions intervenues dans le processus de production et de distribution
antérieur au stade d'imposition (article 2, alinéa 1er, de la première directive);
\item à chaque transaction, la taxe sur la valeur ajoutée, calculée sur le prix du bien ou
du service au taux applicable à ce bien ou à ce service, est exigible, déduction faite
du montant de la taxe sur la valeur ajoutée qui a grevé directement le coût des
divers éléments constitutifs du prix (article 2, alinéa 2, de la première directive).
\end{itemize}
\null
Ces principes furent repris et peaufinés dans la deuxième directive, adoptée à la même date
que la première directive.\\

La deuxième directive définissait déjà les opérations soumises à la taxe, la notion d'assujetti,
les notions de livraison d'un bien, de prestation de services et d'importation. Elle définissait
également la base d'imposition ainsi que l'étendue du droit à déduction.\\

Ces deux directives devaient être en principe transposées dans les différents droits nationaux
pour le 1er janvier 1970 au plus tard, en même temps que les Etats membres devaient abroger
dans leur droit national toute autre taxe sur le chiffre d’affaire.\\

Suite à des difficultés budgétaires et conjoncturelles rencontrées par la Belgique et des
difficultés techniques rencontrées par l'Italie, la date d'entrée en vigueur du 1er janvier 1970
fut reportée au 1er janvier 1972 par la troisième directive7. Le Code de la T.V.A. belge, quant
à lui, entra en vigueur au 1er janvier 1971.\\

La date du 1er janvier 1972 ne put être respectée par l'Italie, ce qui donna lieu à l'adoption des
quatrième et cinquième directives, qui permirent à cet Etat de mettre en application le système
commun de la taxe sur la valeur ajoutée à une date qui n'aura pas été postérieure au 1er janvier
1973.

\section{L'harmonisation progressive de la T.V.A.}

Le système commun de la taxe sur la valeur ajoutée passa un cap décisif lors de l'adoption de
la sixième directive en date du 17 mai 1977.\\

Cette sixième directive abrogea naturellement la seconde directive (article 37 de la sixième
directive), puisqu'elle complète et revoit en profondeur toute une série de matières déjà
réglées par cette dernière.\\

On pourrait s'étonner de la technique juridique qui a consisté en l'adoption d'un nouveau texte
juridique (la sixième directive), plutôt qu’en la modification d'un texte existant (la deuxième
directive).\\

L'explication trouve sa source dans l'importance des modifications apportées au système
commun de la taxe sur la valeur ajoutée, justifiées par plusieurs raisons que l'on retrouve dans
le préambule de la sixième directive, et qui peuvent être résumées comme suit :\\

\begin{itemize}
\item les règles contenues dans la deuxième directive devaient être complétées, afin de
poursuivre la libération effective de la circulation des personnes, des biens, des
services, des capitaux et l'interpénétration des économies;
\item la taxe sur la valeur ajoutée étant partiellement devenue une ressource propre des
Communautés, il convenait de déterminer une assiette uniforme pour l'ensemble
des Etats membres;
\item plusieurs notions, telles celles d'assujetti, d'opération imposable, de lieu des
opérations imposables, de fait générateur et d'exigibilité, etc., devaient être
précisées.
\end{itemize}

\null

Le texte de la sixième directive fut modifié une vingtaine de fois, les modifications
essentielles ayant été nécessitées par la mise en place du régime dit «~transitoire~».\\

Si la sixième directive marqua un pas important dans l'évolution de la taxe sur la valeur
ajoutée, en définissant notamment son assiette uniforme, c'est-à-dire la base sur laquelle elle
doit se calculer12, elle dut toutefois être adaptée à plusieurs reprises aux nouveaux objectifs et
défis que l'Europe se donnait au fur et à mesure des années.\\

C'est ainsi qu'au 1er janvier 1993, l’abolition des frontières entre Etats membres rendait
indispensable la définition d'un nouveau régime T.V.A.\\

Après avoir échoué dans la mise en place de son régime définitif, la Commission
accepta que soit adopté un régime dit «~transitoire~». Aujourd’hui, le régime définitif ne
semble pas prêt d’être adopté, l’Union européenne demeurant accrochée au régime transitoire.

\subsection{NÉCESSITÉ DE DÉFINIR UN NOUVEAU RÉGIME T.V.A. POUR LE 1ER JANVIER 1993}

Les 17 et 28 février 1986, la construction européenne fut marquée par une étape importante : à
ces dates, le Conseil européen approuva les mesures proposées par la Commission dans son
livre blanc, dans le but d'achever le marché intérieur.\\

Ces mesures furent consignées dans «~l'acte unique européen~». Cet acte précise notamment
que l’achèvement du marché intérieur passe désormais nécessairement par une harmonisation
fiscale accentuée ainsi que par l’élimination des frontières fiscales. Les motivations en sont
connues : l’acte répond au souci récurrent d’assurer une meilleure circulation des biens et des
services ainsi que d’éliminer les entraves à la libre concurrence.\\

Pour atteindre les objectifs annoncés, la Commission estimait indispensable d'agir à trois
niveaux :\\

\begin{itemize}
\item l'harmonisation de l'assiette doit être achevée;
\item les taux doivent être alignés;
\item de nouvelles règles applicables aux échanges intracommunautaires doivent être définies lorsque les frontières fiscales auront disparu.
\end{itemize}

\null

La Commission disposait d'environ huit ans pour réaliser les objectifs de l'acte unique,
l'ouverture du marché intérieur étant prévue pour le 1er janvier 1993.\\

Mais pourquoi la suppression des frontières fiscales passait-elle nécessairement par une
révision en profondeur des mécanismes T.V.A. mis en place jusqu'alors ? Un mot
d'explication s'impose.\\

Au niveau européen, on a toujours considéré que le produit de la taxe sur la valeur ajoutée
doit revenir à l'Etat sur le territoire duquel le bien ou le service est consommé.\\

Avant le 1er janvier 1993, dans les transactions internationales (intracommunautaires ou non),
ce principe était respecté de la manière suivante :\\

\begin{itemize}
\item l'exportation de biens (ou de services) était exemptée, étant entendu que le
fournisseur pouvait déduire la T.V.A. qu'il avait lui-même payée en amont, ce qui
rendait le bien (ou le service) exporté «~nettoyé~» de T.V.A;
\item l'importation de biens (ou de services) était taxée, quelque fut le destinataire de ce
bien (ou de ce service).
\end{itemize}

\null

Ce système ne pouvait fonctionner efficacement que si un contrôle de la circulation des
marchandises et des services était exercé aux frontières, le passage matériel de la marchandise
à la frontière constituant le fait générateur de l’importation ou de l’exportation.\\

Or, au 1er janvier 1993, dans le cadre de la réalisation du marché unique, les frontières fiscales
internes à la Communauté devaient disparaître. Ce qui rendait impossible tout contrôle de la
détaxation à l'exportation et celui de la taxation à l'importation. La nouvelle réglementation
T.V.A. devait donc en tenir compte.\\

\subsection{DESCRIPTION DU RÉGIME DÉFINITIF IMAGINÉ PAR LA COMMISSION}

La Commission européenne entama dès lors ses travaux afin de définir des mécanismes
T.V.A. compatibles avec la suppression des frontières fiscales. Le fruit des ses recherches
l’amena à opter logiquement pour une taxation généralisée dans le pays d'origine, c'est-à-dire
dans le pays du fournisseur du bien ou du service.\\

Logiquement, car le territoire communautaire devenant seul et unique, il paraissait naturel
d'élargir aux opérations intracommunautaires le régime applicable aux transactions intérieures
à chaque Etat membre.\\

Or, au sein de chaque Etat, c'est le fournisseur qui est redevable de la T.V.A. envers son Etat.
Par analogie, pensa la Commission, au niveau européen, la taxation doit se faire dans le pays
du fournisseur.\\

Le régime proposé par la Commission (à savoir, la taxation dans le pays d'origine ou,
expression synonyme, dans le pays du fournisseur) pour les opérations intracommunautaires
fut présenté de la manière suivante :\\

\begin{itemize}
\item le lieu de la livraison d'un bien est réputé se situer au lieu de départ de la livraison;
il se situe donc dans le pays d'origine du bien;
\item le fournisseur porte en compte à l'acheteur cette T.V.A. perçue dans le pays
d'origine;
\item la taxe, perçue au taux du pays d’origine, est payée par le fournisseur à son propre
Etat;
\item l'acheteur assujetti établi dans le pays de destination, peut déduire la taxe du pays
d'origine qui lui a été portée en compte; ce droit à déduction est exercé dans son
propre pays;
\item pour respecter le principe d'attribution de la T.V.A. au pays de destination, c’est-à-
dire au pays de consommation, le pays d'origine doit restituer au pays de
destination la taxe encaissée sur l'opération (technique de la compensation ou du
«~clearing~»).

\end{itemize}

Ce système proposé initialement par la Commission supposait une harmonisation des taux.\\

En effet, lorsqu'un consommateur supporte effectivement le poids de la T.V.A., il a
logiquement tendance à acquérir un bien ou un service dans le pays où les taux sont les plus
bas. Dans le système imaginé par la Commission, cette tendance se trouvait renforcée par le
fait que la taxe était perçue dans l'Etat d'origine, au taux applicable dans cet Etat.\\

L'harmonisation des taux (ou, à tout le moins, leur rapprochement) devait donc atténuer les
détournements de trafic et, partant, les distorsions de concurrence.\\

Consciente de cette exigence, la Commission proposa que les Etats n'appliquent que deux
taux différents, à savoir un taux normal (allant de 14 \% à 20 \%) et un taux réduit (allant de
4 \% à 9 \%).

\subsection{L'ADOPTION DU RÉGIME TRANSITOIRE}

Le système de la taxation dans le pays d'origine ne fut pas retenu comme tel, pour deux
raisons essentielles :\\

\begin{itemize}
\item il supposait la modification des structures de taux nationales, ce qui faisait craindre
aux Etats des pertes budgétaires;
\item il supposait la mise en place d'un système de restitution ou de compensation
(système de «~clearing house~») de la taxe perçue par un Etat pour le compte d'un
autre Etat, ce qui paraissait techniquement difficile à réaliser.

\end{itemize}

\null

Le second grief mérite quelques mots d'explication.\\

Comme dit ci-dessus, la taxation dans le pays d'origine et le principe selon lequel le produit
de la taxe doit revenir au pays de destination (ou de consommation), exigent que le pays
d'origine rétrocède la taxe perçue au pays de destination.\\

Bien entendu, vu le nombre élevé d'opérations intracommunautaires, un même pays se
retrouve globalement à la fois pays d'origine et pays de destination à l'égard de tous les autres
pays communautaires, c'est-à-dire à la fois créancier et débiteur des autres Etats membres.\\

D'où l'idée de mettre sur pied un mécanisme de compensation qui permette de régulariser les
flux financiers entre deux pays et de dégager un solde créditeur pour un de ces deux pays.\\

Les difficultés rencontrées lors de l'élaboration d'un tel système paraissaient insurmontables :\\

\begin{itemize}
\item pourquoi percevoir la taxe dans un pays pour la reverser directement à un autre ?
C'est là une source de complications administratives non négligeables;
\item la confiance mutuelle entre les différentes administrations nationales était-elle
suffisamment mûre pour une application correcte de ce système ?
\item en cas de régularisation de taxes opérée dans le pays d'origine (dissimulation du
chiffre d'affaires, etc.), ce dernier perd le bénéfice de la régularisation, puisqu'il
doit en ristourner le profit au pays de destination : cela ne porterait-il pas atteinte à
la motivation et au zèle du fonctionnaire taxateur du pays d'origine ?

\end{itemize}

\null 

Les obstacles des différences des taux et du mécanisme de compensation amenèrent la
Commission à renoncer à l'application immédiate de son système. Elle n'entendit toutefois y
renoncer que provisoirement, dans l'attente d'une plus grande préparation des Etats à son
application.\\

La taxation généralisée et immédiate dans le pays d'origine ayant été (provisoirement)
abandonnée, la Commission s'attela à définir un nouveau régime compatible avec l'abolition
des frontières fiscales au 1er janvier 1993.\\

Ce régime prit naturellement le nom de «~régime transitoire~», par référence au caractère
provisoire de la renonciation de la Commission d'appliquer son système définitif. Ce régime
entra effectivement en vigueur le 1er janvier 1993, suite à l'adoption d'une directive le 16
décembre 1991.\\

Dans le régime transitoire, le principe de l'imposition de la T.V.A. dans l'Etat membre
d'origine fut profondément revu, même s’il est resté l’objectif à moyen terme ou à long terme.
Entre assujettis imposables (les entreprises commerciales et industrielles en général), la
taxation a lieu dans le pays de destination, grâce à la nouvelle technique des «~livraisons-
acquisitions intracommunautaires~».\\

Les opérations entre autres intervenants que les assujettis imposables (les non-assujettis, les
assujettis exemptés, les assujettis franchisés et les assujettis agricoles soumis au régime du
forfait), sont en principe imposées dans le pays d'origine. En principe, seulement, puisque,
dans un souci d'éviter des détournements de trafic uniquement motivés par des considérations
fiscales, des régimes dit «~différenciés~» furent créés. Ces régimes sont différenciés parce
qu'ils ne taxent pas l'opération dans l'Etat membre d'origine mais bien dans celui de
destination.\\

Ces régimes différenciés sont au nombre de trois. Ils ont été définis dans des matières
particulièrement sensibles aux détournements de trafic :\\

\begin{itemize}
\item le régime des ventes à distance (ou par correspondance);
\item le régime des livraisons de moyens de transport neufs, toujours imposées dans
l'Etat membre où le véhicule est immatriculé;
\item le régime des achats par les personnes morales non-assujetties, les assujettis
exonérés, les assujettis franchisés et les assujettis forfaitaires agricoles (la bande
des «~quatre~»), lorsque le montant de ces achats dépasse un certain seuil.

\end{itemize}

\null

Ces régimes différenciés sont d'une rare complexité et donnent lieu à de nombreuses
difficultés d'application. Cette complexité est incontestablement le talon d'Achille du régime
transitoire.\\

La complexité du régime transitoire fut reconnue implicitement par le législateur européen qui
adopta deux directives dites de «~simplification~», datées respectivement des 14 décembre
1992 et 10 avril 1995.\\

L'adoption du régime transitoire fut entourée de nouvelles mesures en matière de taux,
reprises dans une directive adoptée le 19 octobre 1992.\\

Depuis le 1er janvier 1993, les Etats membres doivent appliquer un taux normal et un ou deux
taux réduits. Le taux normal ne peut être inférieur à 15 \%, le(s) taux réduit(s) ne pouvant être
inférieur (s) à 5 \%. Etant entendu que c’est la sixième directive qui détermine quels biens et
services peuvent être soumis au(x) taux réduit(s), les autres biens et services devant être
soumis au taux normal (annexe H de la sixième directive).\\

\subsection{VERS LE RÉGIME DÉFINITIF}

Lors de l'adoption de la directive du 16 décembre 1991, il était prévu que le régime transitoire
n'entre en vigueur que pour une période limitée de quatre ans, étant entendu qu'au 31
décembre 1996, s'appliquerait un nouveau régime, définitif celui-là, reposant sur le principe
tout à fait généralisé de l'imposition dans l'Etat membre d'origine des biens livrés et des
services rendus, accompagnée d'un mécanisme de compensation entre Etats membres et d'une
(plus grande) harmonisation des taux.\\

Le préambule de la directive du 16 décembre 1991 fixait d'ailleurs l'agenda suivant : la
Commission présente au Conseil, avant le 31 décembre 1994, un rapport sur le
fonctionnement du régime transitoire et des propositions sur les modalités du régime définitif,
pour permettre d'adopter les bases du régime définitif (sous la forme de directive(s)) avant le
31 décembre 1995.\\

La Commission a accusé un certain retard dans son travail, étant manifestement plus occupée
à rechercher des solutions aux (nombreuses) difficultés causées par l'application du régime
transitoire que par la préparation d'un (nouveau) régime définitif.\\	

A l'heure où ces lignes sont écrites, peu d'informations ont filtré sur les travaux d'avancement
de la Commission quant à l’adoption du régime définitif. Un moment, on a cru que le régime
définitif n’entrerait en vigueur qu’en l’an 2000.\\

Aujourd’hui, il semble bien que la Commission éprouve toutes les difficultés pour mettre au
point une proposition de directive devant établir le régime définitif, difficultés qui trouveraient leur source dans la relative inertie des Etats membres ou, pire, dans leur souci de
ne pas perdre un peu plus de leur souveraineté. A cet égard, la règle de l’unanimité en
vigueur au niveau européen pour adopter de nouvelles directives semble être une pierre
d’achoppement fort importante.\\

\section{ L’adoption de la nouvelle directive T.V.A.}

La sixième directive ayant été modifiée à de très nombreuses reprises, l’Union européenne a
décidé, dans un souci de clarté et de rationalité, de procéder à la refonte de la sixième
directive.\\

Ce fut fait le 28 novembre 2006 par l’adoption de la directive 2006/112/CE relative au
système commun de taxe sur la valeur ajoutée, directive dite «~directive T.V.A.~». Cette
directive, composée de 414 articles, n’apporte en principe aucune règle supplémentaire par
rapport aux textes antérieurs, son adoption n’étant motivée que par un souci de codification et
de clarification des règles existantes.\\

Dans la logique de sa rédaction, la directive T.V.A. abroge la première et la sixième directive.\\

\section{Influence de la Cour de justice des Communautés européennes dans l’harmonisation de la taxe sur la valeur ajoutée}

Comme on le sait, la Cour de justice des Communautés européennes «~assure le respect du
droit dans l’interprétation et l’application du présent traité~».\\

Depuis l’entrée en vigueur du régime de la taxe sur la valeur ajoutée, l’importance de la Cour
de justice en cette matière ne s’est jamais démentie. Elle rend en la matière entre vingt et
trente arrêts par an.\\

La richesse de la jurisprudence de la Cour trouve notamment sa cause dans l’application
immédiate des directives T.V.A. Cet application immédiate autorise les justiciables à
invoquer devant leurs juridictions nationales l’argument de la violation du droit
communautaire par leur droit national, ce qui conduit alors aux mêmes juridictions nationales
de poser à la Cour de justice des questions préjudicielles.\\

En substance, les arrêts de la Cour ont pour effet d’assurer une cohérence et une application
harmonisée des réglementations européennes (principalement les première et sixième
directives), à l’encontre de réglementations nationales qui divergeraient des objectifs
poursuivis.\\

A l’heure actuelle, aucun praticien ne peut pratiquer valablement la taxe sur la valeur ajoutée
sans se référer systématiquement, dans les dossiers qu’il gère, à la réglementation européenne
et plus particulièrement, à la jurisprudence de la Cour.\\

L’importance de la jurisprudence de la Cour de justice des Communautés européennes
devient d’autant plus grande que l’Union européenne semble quelque peu en panne pour
adopter de nouvelles mesures en droit positif.\\

\chapter{L'introduction de la taxe sur la valeur ajoutée en droit positif belge}

L'introduction du Code de la taxe sur la valeur ajoutée en droit positif belge est due à la loi du
3 juillet 1969. Suite à diverses difficultés conjoncturelles et budgétaires rencontrées par la
Belgique, ce Code n'est entré en vigueur qu'au 1er janvier 1971.\\

Avec la cinquantaine d'arrêtés royaux qui le complètent, le Code de la T.V.A. est le texte
juridique de base du système belge de la taxe sur la valeur ajoutée.\\

Ces textes ont été modifiés à de nombreuses reprises, pour des raisons qui tiennent
essentiellement à l'évolution européenne en la matière.\\

La loi du 28 décembre 1992 revêt à cet égard une importance particulière puisqu'elle introduit
dans le droit positif belge la sixième directive européenne ainsi que le régime transitoire.

\chapter{Les caractéristiques de la taxe sur la valeur ajoutée}

Les caractéristiques de la T.V.A. peuvent être facilement illustrées par un exemple.
Cinq intervenants : un bûcheron, un menuisier, un ébéniste, un revendeur et un client final. Le
taux appliqué à l'exemple est un taux uniforme de 21 \%.\\

\begin{tabular}{|c|c|c|c|c|}
  \hline
   & Prix de vente & TVA due & TVA à déduire & Solde \\
  \hline
  	Bûcheron & 1000 & 210 & - & 210\\
    Menuisier & 1200 & 252 & 210 & 42\\
    Ebéniste & 1400 & 294 & 252 & 42\\
    Revendeur & 1600 & 336 & 294 & 42\\
    Client & - & - & - & -\\
     \hline

Total & - & 1092 & 756 & 336\\

  \hline
\end{tabular}

\null

Le total de la cinquième colonne indique le montant de la T.V.A. qui revient en définitive à
l'Etat.\\

Identifions les caractéristiques de la T.V.A. à travers cet exemple.

\section{La T.V.A. est un impôt sur le chiffre d'affaires}

L'exemple révèle le caractère de «~taxe sur le chiffre d'affaires~» de la T.V.A. En effet, elle
est perçue de manière directement proportionnelle sur le chiffre des ventes faites par les
différents intervenants.\\

En vertu des première et sixième directives T.V.A., la taxe sur la valeur ajoutée est la seule
taxe sur le chiffre d'affaires qui puisse être en vigueur sur le territoire des Etats membres.
L’article 401 de la directive T.V.A. est ainsi rédigé :\\

\textit{«~Sans préjudice d’autres dispositions communautaires, les dispositions de la présente
directive ne font pas obstacle au maintien ou à l’introduction par un Etat membre de taxes sur
les contrats d’assurance et sur les jeux et paris, d’accises, de droits d’enregistrement, et, plus
généralement, de tous impôts, droits et taxes n’ayant pas le caractère de taxes sur le chiffre
d’affaires, à condition que la perception de ces impôts, droits et taxes ne donne pas lieu dans
les échanges entre Etats membres à des formalités au passage d’une frontière~».}\\

Pour apprécier si une taxe, un impôt ou un quelconque prélèvement présente les
caractéristiques d'une taxe sur le chiffre d'affaires, la Cour de justice des Communautés
européennes retient trois conditions essentielles:\\

\begin{itemize}
\item la taxe s'applique de manière générale aux transactions ayant pour objet des biens
ou des services;
\item elle est toujours parfaitement proportionnelle au prix payé par le client;
\item elle est perçue à chaque stade du processus de production et de distribution.
\end{itemize}

\null 
Toute taxe, impôt ou prélèvement qui remplirait simultanément ces trois caractéristiques doit
être qualifié de taxe sur le chiffre d'affaires.\\

La Cour de justice des Communautés européennes a également eu l'occasion de préciser que,
pour apprécier si une taxe a le caractère de taxe sur le chiffre d'affaires, il y a notamment lieu
de vérifier si elle a pour effet de compromettre le fonctionnement du système commun de
T.V.A. en grevant la circulation des biens et des services et en frappant les transactions
commerciales d'une façon comparable à celle qui caractérise la T.V.A. (affaire Gabriel
Bergandi).\\

Ainsi, selon cette même Cour, ne peut être considérée comme une taxe ayant le caractère de
taxe sur le chiffre d'affaires une taxe qui, quoique comportant des montants différents selon
les caractéristiques du bien imposé, est assise sur la seule mise à disposition au public du bien,
sans considération effective des recettes pouvant être réalisées par cette mise à disposition.\\

\section{La T.V.A. est un impôt à la consommation}

La T.V.A. est un impôt à la consommation, ce qui implique qu'en principe, seul le
consommateur final doit en supporter la charge.\\

Dans l'exemple, c'est en effet bien le consommateur final qui supporte la charge totale de la
taxe, puisqu'il paie 21 \% du prix payé à son revendeur, sans aucune possibilité de déduction.\\

On tire généralement de cette caractéristique la conséquence suivante : la T.V.A. étant
finalement supportée par le consommateur final, son produit doit revenir, en totalité, au pays
de consommation des biens ou des services, au taux de ce pays.\\

Dire que la taxe sur la valeur ajoutée est un impôt à la consommation implique également que
l'impôt ne peut être perçu s'il n'y a pas consommation. Il s'agit là d'un enseignement de la
Cour européenne de justice, notamment tiré d'un arrêt du 29 février 1996.\\

\section{La T.V.A. est acquittée par paiements fractionnés}

Comme son nom l'indique, la T.V.A. est perçue sur la valeur ajoutée, à chaque étape du
circuit économique d'un bien ou d'un service, étant entendu que c'est le fournisseur qui est en
principe redevable de la taxe à l'égard de l'Etat, après déduction des taxes qu'il a payées à ses
propres fournisseurs.\\

Chaque assujetti devient par conséquent un collecteur d'impôt, à l'endroit où il se situe dans le
circuit économique. Il collecte une partie de la T.V.A. totale que supportera le consommateur
final.\\

Confrontée à l'exemple, cette caractéristique signifie que les 336,00 de taxe que supporte le
consommateur final sont perçus par l'Etat à cinq moments différents (210 + 42 + 42 +
42), à chaque étape du circuit économique du bien vendu.\\

La T.V.A. est donc payée par «~fractionnement~» puisque l'Etat perçoit la taxe en quatre
moments différents. Le temps s’écoulant entre ces quatre moments dépend du secteur
économique dans lequel les opérations s’inscrivent, mais aucune contrainte à cet égard
n’existe.\\

\section{La T.V.A. est neutre et ne fait pas partie du prix de revient}

La neutralité de la T.V.A. présente une double facette : une facette verticale et une facette
horizontale.\\

Verticalement, la neutralité implique que le montant de la T.V.A. due à l'Etat n'est pas
influencé par la longueur du circuit économique qui précède la consommation finale,
contrairement aux systèmes de taxes cumulatives en cascade. La neutralité verticale trouve sa
source à l’article 1er, paragraphe 2, de la directive T.V.A. ainsi rédigé :\\

«\textit{ Le principe du système commun de T.V.A. est d’appliquer aux biens et aux services un
impôt général sur la consommation exactement proportionnel au prix des biens et des
services, quel que soit le nombre des opérations intervenues dans le processus de production
et de distribution antérieur au stade d’imposition.}\\

\textit{A chaque opération, la T.V.A. calculée sur le prix du bien ou du service au taux applicable à
ce bien ou à ce service, est exigible déduction faite du montant de la taxe qui a grevé
directement le coût des divers éléments constitutifs du prix.}\\

\textit{Le système commun de T.V.A. est appliqué jusqu’au stade du commerce de détail inclus.}»\\

La neutralité (verticale) de la T.V.A. est garantie par la technique de la déduction, qui permet
à chaque intervenant précédant le consommateur final de déduire de la taxe due à l'Etat la taxe
qu'il a lui-même payée à son cocontractant. Ce droit à déduction porte sur la T.V.A. afférente
à tous les biens et services qui ont contribué directement et immédiatement à la réalisation
d'opérations à la sortie imposables en principe (voyez l'article 45, § 1er, du Code de la
T.V.A.).\\

La neutralité (verticale) de la T.V.A. est essentielle. Elle évite que la vie économique se
développe essentiellement sur la base de considérations de fiscalité indirecte plutôt que
d'impératifs industriels, économiques ou commerciaux. A prix final égal, T.V.A. égale,
quelle que soit la longueur du circuit.\\

La déduction de la taxe payée en amont a également pour conséquence que cette dernière n'a
en principe pas d'influence sur le prix de revient des biens et des services acquis par des
assujettis.\\

La neutralité horizontale, quant à elle, découle de la jurisprudence de la Cour de justice des
Communautés européennes qui a précisé que la neutralité de la T.V.A. implique qu’en
principe, toutes les activités économiques de même nature doivent être traitées de la même
manière.\\

Cette neutralité horizontale impliquerait donc que des personnes différentes effectuant des
opérations similaires doivent subir le même régime T.V.A.\\

Il s’agit là d’une jurisprudence saine qui renforce la cohérence du système T.V.A., même si
cette horizontalité connaît également, à l’instar du principe de neutralité verticale, des
entorses, essentiellement provoquées par le régime des exemptions établi à l’article 44 du
Code de la T.V.A.\\

\section{La T.V.A. est transparente}

Quand un agent économique acquiert un bien sous le régime de la T.V.A., il connaît
exactement la charge fiscale qui grève le bien acquis.\\

Cette caractéristique est généralement retenue pour souligner la différence entre le mécanisme
de la T.V.A. et le système des taxes cumulatives en cascade, dans lequel la charge fiscale
grevant une marchandise peut être inconnue.\\

\chapter{TVA et notion d'assujetti}

La TVA frappe des opérations : livraison de biens, prestation de service, Importation et AIC.\\	

Assujettis :
\begin{itemize}
\item Assujetti imposable
\item Assujetti exonéré
\item Assujetti franchisé
\item Assujetti forfaitaire agricole
\end{itemize}

Non assujettis ;
\begin{itemize}
\item consommateur privé
\item personne morales non assujetties (autorité publique, holding passif)
\end{itemize}

\null

Notion assujetti : Article 4 CTVA, Article 9 Directive. Attention pas de lien entre l'assujettissement et la personnalité juridique (ex : société momentannée et unité TVA).\\

Article 9 - Directive : notion d'activité économique.\\
Activité habituelle -> Doctrine TVA : Répétition et régularité dans l'accomplissement.\\

Tribunal de Namur. 06.02.2008 -> Recette (bénéfice/perte)\\

Assujetti même si echec de l'entreprise avant sa création.\\

Début : Intention non équivoque.\\

Fin : en cas de faiilite -> Reprise par le curateur. Pour ne pas créer de concurrence déloyale. Les opération terminatoires peuvent donner lieu à déduction.\\

Vendeur batiment neuf : article 8\\

Livraison de transport neuf : article 8bis - Déduction 45 §1er non limité par le 45 §2.\\

Régime différencié :\\

\begin{itemize}
\item Ventes à distance
\item Vente Moyen de Transport Neuf
\item Vente à la bande des 4
\end{itemize}

\null

Article 44 §1 - Personne nommée\\
Article 44 §2 - Prestations d'intérêt général.\\
Article 44 §3 - Divers\\

Exonération d'interprétation restrictive.\\

Arrêt CJE 01.12.2005 -> Service téléphonique , .... par les hôpitaux ne constituent pas des opérations exemptées sauf si ces prestations revet une aspect indispensable. La distortion de concurrence justifie la position de la CJE.\\

Article 44, paragraphe 2, 4èmement -> Enseignement et recyclage professionnel - Circulaire 25 dd. 24.12.1993 annulée par l'arrêt  145.138 du 30 mai 2005 du Conseil d'Etat pour violation de l'article 172, alinéa 2 de la Constitution. \\

Article 44, paragraphe 2, 12emement -> Kermesse et fancy-fair\\

Article 44, paragraphe 2, 13emement -> Revente par un assujetti exempte\\

Article 44, §3, 1 :\\
\begin{tabular}{|c|c|c|c|}
  \hline
   & Promoteur immobilier & Cédant occasionnel avec activité économique & Cédent occasionnel sans activité économique \\
  \hline
  	Comptable & Stock & Immobilisation corporelle & Patrimoine privé\\
     \hline    
    Droit à déduction & Immédiat & Selon son droit à déduction / diféré si article 44 & Différé\\
     \hline   
    Prélèvement (non-vente dans les 2 ans) & Article 12, §2 & Rien & Rien\\
     \hline
    Option & Non & Oui & Oui\\
     \hline
    Révision & Non & Oui sauf si article 44 & Non\\
  \hline
\end{tabular}

\null

Les notions de TVA ne sont pas interprétable par rapport au droit nation. La référence au droit interne n'est pas pertinente pour appréhender pour l'exemption applicable à la location. \\

Les exonérations prévues par la 6ème directive constituent des notions autonomes du droit communautaire et doivent dès lors recevoir une définition communautaire.\\

En d’autres termes, les diverses autres législations nationales ne sauraient faire obstacle à l’application
uniforme de la T.V.A. dans les différents Etats membres (droit civil, baux commerciaux, à ferme, etc.).\\



CJCE - 6 éléments :

\begin{itemize}
\item un immeuble (devant etre prépondérant de la prestation)
\item un propriétaire bailleur
\item un locatire
\item une cession du droit d'utiliser du bien et de pouvoir en exclure quelqu'un (droit exclusif)
\item une durée convenue
\item un loyer
\end{itemize}

Arret Seeling
Arret Temco - 18.11.2004

TEMCO est une société qui exerce une activité de nettoyage et d’entretien de bâtiments. Elle est propriétaire d’un immeuble dans lequel elle n’a pas de siège d’exploitation. TEMCO conclut, avec trois sociétés, trois conventions individuelles qualifiées par les parties comme des cessions. Ces conventions autorisent chaque occupant à exercer ses activités dans l’immeuble, selon une affectation décidée par TEMCO.\\

Les quatre sociétés font partie du même groupe financier (sociétés liées). Les occupants n’ont aucun droit particulier sur l’immeuble.\\ 

Les conventions sont conclues pour la durée des activités des occupants, lesquels sont tenus d’utiliser les locaux exclusivement pour la réalisation de leur objet social. TEMCO peut, à tout moment et sans préavis, exiger la libération de l’espace concédé (occupation précaire). Le prix est essentiellement fixé par rapport au nombre
de mètres carrés occupés, augmenté d’un pourcentage du chiffre d’affaires et d’un forfait par personne occupée.
Les parties ont expressément exclu le bail immobilier.

Chacun des occupants dispose d’une clé d’accès au bâtiment et est soumis au règlement d’ordre intérieur pour l’accès, le nettoyage, le raccordement au téléphone, etc. L’immeuble est sis à Bruxelles.

La location d’immeuble constitue normalement :
\begin{itemize}
\item une activité relativement passive ;
\item liée au simple écoulement du temps ;
\item et qui ne génère pas une valeur ajoutée significative.
\end{itemize}
\null
par rapport à d’autres activités qui ont :\\

\begin{itemize}
\item soit un caractère d’affaires industrielle et commerciale, telles les activités visées par la non-exonération (parkings, hôtels, campings, etc.)
\item soit un objet mieux caractérisé par l’exécution d’une autre prestation que la simple mise à disposition d’un bien (p. ex. droit d’utiliser un terrain de golf, un pont, d’installer des distributeurs de cigarettes dans un établissement commercial).
\end{itemize}
\null
En substance, la Cour arrête qu’est une location immobilière, la convention par laquelle :\\
\begin{itemize}
\item une société octroie simultanément, par des contrats différents, à des sociétés qui lui sont liées
\item un droit précaire d’occupation sur le même immeuble
\item contre le paiement d’une indemnité fixée principalement en fonction de la surface occupée
\item lorsque ces contrats, tels qu’ils sont exécutés, ont essentiellement pour objet la mise à disposition passive de locaux ou de surfaces d’immeubles
\item moyennant une rémunération liée à l’écoulement du temps et non une prestation de services susceptible de recevoir une autre qualification.
\end{itemize}
\null




   
\nocite{*}
\bibliographystyle{plain}
\bibliography{publications2}       % 'publications' is the name of a BibTeX file
\addcontentsline{toc}{chapter}{Bibliographie} 
\end{document}