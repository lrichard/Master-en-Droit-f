\documentclass{book}
\usepackage[utf8]{inputenc}
\usepackage[T1]{fontenc}
\usepackage{lmodern}
\usepackage[a4paper,left=3.5cm,right=2.5cm,top=2.5cm,bottom=2cm]{geometry}
\usepackage[frenchb]{babel}
\usepackage{fncychap}
\usepackage{fancyhdr}
\usepackage{sectsty}
\usepackage[pdftitle={Master complémentaire en droit fiscal - Droit fiscal international et comparé},pdfauthor={Laurent RICHARD}, pdfsubject={Notes de cours}, pdfkeywords={ULg, Fisc}]{hyperref}
\usepackage{graphicx}
\usepackage{pdfpages}
\usepackage{listings}
\usepackage{lscape}
\allsectionsfont{\sffamily}
\usepackage{fancybox}
\setcounter{secnumdepth}{4}
\usepackage{latexsym}
\usepackage{pifont}
\renewcommand\FrenchLabelItem{\textbullet}
\widowpenalty=9999
\clubpenalty=9999




\begin{document}
\sffamily
\newcommand{\RPoint}{\protect\includegraphics[height=1.7ex,keepaspectratio]{point.png}}
\newcommand{\RSave}{\protect\includegraphics[height=1.7ex,keepaspectratio]{Save.png}}
\renewcommand\labelitemii{\ding{220}}
\begin{titlepage}

\begin{center}
\begin{Large}ULg - Master complémentaire en droit fiscal\end{Large}\\
\vspace{0.5cm}- \\
\vspace{0.5cm}
Fiscalité des entreprises

\end{center}
\vspace{6cm}

\LARGE
\begin{center}
\textsc{Notes de cours - Droit fiscal international et comparé}\\
\end{center}

\vspace{10.0cm}

\normalsize
\flushright{
\textbf{Laurent RICHARD}\\
Comptable-fiscaliste agréé IPCF\\
\vspace{0.5cm}
Année académique 2012-2013}

\end{titlepage}

\newpage

\thispagestyle{empty}
\setcounter{page}{0}
\null

\newpage
\thispagestyle{empty}
\setcounter{page}{0}
\vspace{20cm}

\vfill
\begin{flushright}
Life is a succession of lessons, \\
which must be lived to be understood. \\ 
--- \textit{Ralph Waldo Emerson}
\end{flushright}
\vfill
\newpage

\renewcommand{\chaptermark}[1]{\markboth{#1}{}}
\renewcommand{\sectionmark}[1]{\markright{\thesection\ #1}}
\fancyhf{} \fancyhead[LE,RO]{\bfseries\thepage}
\fancyhead[LO]{\bfseries\rightmark}
\fancyhead[RE]{\bfseries\leftmark}
\renewcommand{\headrulewidth}{0.5pt}
\addtolength{\headheight}{0.5pt}
\renewcommand{\footrulewidth}{0pt}
\fancypagestyle{plain}{ \fancyhead{}
\renewcommand{\headrulewidth}{0pt}}

\thispagestyle{empty}
\setcounter{page}{0}
\null
\newpage
\pagenumbering{roman} \setcounter{page}{1} 




\tableofcontents



\chapter{Cours du 09.02.2013}
\pagenumbering{arabic} \setcounter{page}{1} 

\section{Introduction}

\chapter{Cours du 23.02.2013}

Test

C’est le SPF Finances qui négocie les CPDI. Les loi d’approbation par le SPF Affaires étrangères.

Une fois que la loi est votée, il faut un échange de documents de ratification (preuve que tout est bien accepté). Puis seulement publication au Moniteur Belge.

La date d’entrée en vigueur et de prise d’effet sont deux choses différentes. On essaye de mettre le 1er janvier ou le début d’un exercice. Il est possible que la prise d’effet soit antérieur à la date d’entrée en vigeur. Cass.

Exemple : Suède (signée en 1965, approuvée en 1967 et ratifiée en 1967). Un contibuable demande application pour 1966. Cassation dit non car pas entrée en vigeur.

UK signée en 1967, approuvée 1970 et ratifiée 1970 mais effet pour exercice1967. Cassation dit non pour revenus 1966. Cela semble erronée car les traités prime sur le droit belge.

Allemagne signée en 04.1967, approuvée en 1969 et ratifiée en 1969 avec aussi effet exercice 1967. Mais ici, dans la loi d’approbation, on prévoit une procédure pour les régularisation. Cass accepte.

USA prévoyant une diminution des précomptes sur dividendes a été signé en 1987, approuvée 1989 et ratifiée en 1989 mais prévoyant l’application à partir de 1988. Deux pratiques (payé tarif plein et réclamation après OU certains appliquent directement le taux réduit).

Le précompte n’est pas un impôt pour le débiteur mais bien pour le bénéficiaire. Lorsque la société belge fait un recours, il n’y a pas d’intérêt moratoire. Bien si c’est l’entreprise US.

Convention avec France. 5 protocoles dont un en 1999 avec des nouvelles règles de non discriminations. Effet en 1996 ! Quand on signe, on sait qu’elle va reculer de 3 ans. Personne ne sait donc en 1996 que l’on va signer un protocole en 1999. Délais de réclamation épuisé. Approuvée en 1999 et ratifiée en 2000. On a réouvert les délais de réclamation jusqu’en 2001.

Quand on est plus d’accord, on peut dénoncer le traité (sauf les USA qui modifie uniquement en interne). On renégocie et le suivant remplace l’autre à la date de prise d’effet (cfr nouvelle convention Pays-Bas et USA). Cas exeptionnel de jurisprudence concernant une succession d’Etat (ancienne convention avec la Russie qui est applicable avec les anciens pays soviétiques avec lesquels il n’y a pas encore de convention). Signée en 1990. URSS explose en 1991. Société belge ayant des activité en Ouzbekistan (1996). Accord en 1992 avec l’Ouzbekistant pour autant qu’elle ne nuise pas aux deux états, d’appliquer l’ancienne convention russe. L’ouzbekistan prélève un impot. L’Etat belge va voir l’ouzbekistan qui dit qu’il n’y a pas de convention et propose de négocier une nouvelle avec prise d’effet en 1996. Mais la société ne récupère que la moitié des impots payés. Action en responsabilité contre l’Etat belge. Arrêt de 2010 de la Cass. qui dit qu’en cas de succession, le successeur doit respecter les anciennes conventions. Cfr donc URSS et Yougoslavie. La Belgique les applique donc unilatéralement.

La Convention de Vienne s’applicque aux convention depus 1990 mais vu qu’elle transcrit la coutume, cela s’applique aussi à celle avant.

Principe d’application de bonne foi. Reconstituer la volonté des parties de manière unilatérale
Ex : Article 364bis - Si déduction des primes de pension, l’idée est la taxation à la pension. Certains sont partis à l’étranger juste avant la pension. Taxation la veille du départ (entrave à la libre circulation). Contraire au principe de bonne foi de taxer avant le départ alors qu’il pourrait être taxable dans le nouveau pays ?

Interprétation bilatérale. Il faut faire un effort pour cela. Cass. en matière d’application des salaires dans la CPDI avec le Luxembourg. Cass considère dans le cas des transports internatiaux, ce n’est pas où on est qu’on est taxé et donc un travailleur d’une société luxembourgeoise est taxé au siège de l’employeur. (cfr article 8)

Protocole additional avec le Lux : Extention de l’article 8 de la convention au transport routier et taxation où on exerce.
Appel Mons disant que le rajout est une confimration de l’idée des partie, c’est donc interpétatif donc aussi rétro-actif. Cas cours fiscal canadienne 22.04.2008 prévosicar.

Interprétation littérale : Ce qui n’est pas écrit, n’est pas écrit. On ne peut pas présumer qu’un état a abandonné sa souveraineté. Cfr article 25 §3 OCDE possibilité de discution. Pas d’application dans le modèle belge car on ne peut discuter sur une convention négociée par le Roi. 

Cass interprétant seule le droit français. France-Belgique.

Interprétation téléologique. Interprétation selon les objectifs.

Interprétation systématqiue. Selon le contexte partout dans le texte. Un seul sens.  Contexte : textes bilatéraux. En pratique, l’exposé des motivé est utilisé vu qu’il est rédigé par le fonctionnaire ayant négocié. Convention BE-NL, circulaire interprétative bilatérale. Circulaire jointe aux travaux préparatoires. Le commentaire OCDE vaut comme contexte. Si un texte n’a pas de sens, il faut considéré que les Etats n’ont rien convenu.

FR-BE (1964) Régime des travailleurs frontaliers. Porteur de la carte frontallière. 1965, fini la carte frontalière car symbole distinctif. Donc, les frontaliers sont ceux qui auraient la carte. Cass. dit que si pas de carte, pas de frontalier. Application néanmois, régime optionnel.

Principe d’interprétation OCDE
Chapite II article 3 4 5 -> Définition un peu partout. Ce n’est pas nécessairement la même définition partout.

Certaines ne servent pas à grand chose. Entreprise d’un état contractant (article 3 1. d)
Quid du terme entreprise. (cfr article 14 dans le modèle -> supprimé mais +/- article 7). Plutôt que de modifier le titre de l’artcile 7, on a ajouté la définition d’entreprise. article 3 1.h intègre l’article 14 dans le 7. Le salaire n’est pas défini -> Article 3 §2, interprétation par le droit national de l’etat qui applique la convention. Indemnité de dédit -> Interprétation du droit FISCAL. Le commentaire de l’artile 12 renvoit vers d’autre interprétantion.


Les personnes visées
-------------------------

Article 1er. 

Non discrimination pour les non-nationaux ainsi que les disposition d’entraide administrative.

ARt 1er visent les résident d’au moins une des états contractant (PP, société et groupement de personnes).

Société : Personne morale y compris les ASBL. Traité comme un seul contribuable. 

En Belgique, les associés sont censé réalisés eux-même les bénéfice (rapartition selon leur part (sauf conevntion en part virile).

INR, 228 secundo, les société non-personnifiées , qui n’ont pas leur établissement principal en Belgique. sont considérés comme des sociétés. Mais il faut être considéré comme un résident (pas uniquement comme un contribuable) pour pouvoir appliquer une convention. La nationalité est un critère personnel et non territorial. L’OCDE vise les critères territoriaux dans son article 4. Un américain paye l’IRS car il est américain. La convention avec les USA diffère du modèle OCDE pour cette raison.

Certains résidents proviennent des pays off-shore qui ne taxe que les pays ne taxant que les revenus nationaux. Exclusions des pays off-shore pour l’application des conventions.

On a voulu visé les off-shores agressifs.

Question des fonds de pension.
-----

Généralement en exemption d’impôts. Ceux-là ne sont pas taxés de manière suffisante pour être intégré via l’article 4 sauf si inclus de manière explicite. Cfr 2e protocole additionnel.


Pour que les conventions fonctionne, un faiut une seule résidnece. Donc tie-breaker. 4 §2 et 4 §3.

Foyer d’habitation permanent (dispositin permamnente et non habitation permanente).Centre d’intérêts vitaux. Puis séjour habituel (183 jours) puis nationalité puis arrangement des états.

Pour les sociétés, il s’agit du siège de direction effectif. La cour de Cassation a jugé qu’il est possible d’être résident belge car on gère sa fortune en Belgique même si on n’y est pas domicilié. Cfr Registre de la population obligatoire après 3 mois. Présomption réfrageable. Circ. 08.08.1983 (Régime d’imposition pour les cadres internationaux) Pour les personnes non résidents engagés par une firme internationale. Possibilité de déductions de forfaits de coûts de logement, de détachement, de différence d’imposition et l’abattement va jusqu’à 30.000 EUR. Taxation sur une portion de cette base à savoir sur le pro-rata des jours de présence en Belgique. Pas de trace dans le CIR92. Suite à la circulaire en 76 qui a été jugée illégale.

Présomption irréfrageable si installation en Belgique du ménage -> Résident belge.

Si une société rate l’application de l’article 4 §3 -> elle n’est de-facto pas assujettie à l’ISOC mais à l’INR-SOC. -> Risque de perte de convention tierces.

Autre effet 4 §3, dans la directive mère-filiale, il est prévu que la directive ne s’applique pas aux société à double nationalité et résidente d’un pays tiers. 


Ex ancien : société double résidence Pays-Bas et Antilles néerlandaises.. Si résident des Antilles, pas de taxation des P-B. Si dividendes Belgique - Pays-Bas, pas d’exonération sur base de la Directive et pas application de la convention B- NL car on perd la résidence NL.

Taxation à forfait en Suisse. On ne peut pas travailler en Suisse. Taxation sur les dépenses effectuées avec une base minimum de dépense. Mais pas accès aux conventions. La Belgqiue n’est pas obligé de leur accorder les exonérations conventionnelles. Si on veut bénéficier de la convention, il faut que la Suisse taxe les revenus complémentaires.

Taxation UK : Taxation sur les revenus rapatriés. Donc, il ne faut pas rapatrier. Pas allégements conventionnels de la Belgqiue si les revenus ne sont aps rappatrier. l’article 4 est un tie-breaker rule.

Dans ceratines conventions, il n’y a pas de tie-breaker, la convention ne s’appilque pas mais application de l’article 23.

Deux pays peuvent interprété de manière différente la catégorisation des revenus -> Conflit. LEX FORI. John Everjones, Klaus FOGEL, ...

Everyjones lit le 3 §2. Celui qui lit la convention fait sauté un verrou. Celui qui est source donne la définition.
Selon Fogel, le pays source impose aau pays résidence ce qui est anormal. On planche donc sur la question des entités hybrides. Groupement considéré comme résident d’un état mais pas de l’autre. Si le pays source ne reconnait pas la société dans le pays résidence et considère que les associés sont dans redevables directement au pays source. -> Conflit. Demande d’attestation de résidence. Cela intervients urtout dans la problématique des plus value sur les parts. Taxable pays source ou résidence. Possibilité de double déduction/exonération/imposition.

Rapport OCDE sur les société de personnes. Il se base sur les thèse de Everyjones et intégré dans le commentaire d’Everyjones. Mais en pratique, on se réfère au droit national.

Mais si le pays source donne la définition, c’est bien mais il manque l’obligation au pays résidence d’accepter cette définition.

Circulaire en Belgqiue qui transpose cela. Exemption en Belgique si imposition dans le pays source conformément à la convention. Concernant le remède à l’imposition, c’est les méthodes belges et donc la qualification belge. (Dividende vs Revenus immobiliers).

Si changement dans la législation national, l’interprétation suit (interprétation au moment) sauf si contexte.

Cfr notion de dividende provenant d'action. En droit FR, c'est une somme votée par une A.G.

Quid des fictions. Article 36 CIR92. Requalification des loyers en revenus des dirigeants d'entreprise. Il serait anormal que la Belgique perde un pouvoir d'imposition ou de la mise en place d'une mesure anti-abus. Il faut tenir compte du commentaire OCDE. Le modèle ne change pas mais le commentaire bien. L'OCDE veut qu'on utiliser le commentaire le plus récent même pour les anciennes conventions. En effet, le commentaire de l'époque a servi aux conventions. La jurisprudence belge (sauf accident) utilise le commentaire de l'époque. De même cela semble incorrecte de prendre le commentaire de 64 pour la conevntion FR-BE qui n'est pas basé sur le modèle OCDE final (utilisation d'un brouillon).

Devoir :

Etudier l'article 5 - Définition établissement stable
Article 7
Article 9
 





\nocite{*}
\bibliographystyle{plain}
\bibliography{publications2}       % 'publications' is the name of a BibTeX file
\addcontentsline{toc}{chapter}{Bibliographie} 
\end{document}
