\documentclass[12pt]{report}
\usepackage[utf8]{inputenc}
\usepackage{mathptmx}
\usepackage[T1]{fontenc}
\usepackage{lmodern}
\usepackage[a4paper,left=3.5cm,right=1.5cm,top=3.0cm,bottom=2.5cm]{geometry}
\usepackage[frenchb]{babel}
\usepackage{fncychap}
\usepackage{fancyhdr}
\usepackage{sectsty}
\usepackage[pdftitle={Titre du TFE},pdfauthor={Laurent RICHARD}, pdfsubject={Travail de fin d'études - Master complémentaire en Droit fiscal}, pdfkeywords={Laurent, RICHARD, TVA, Tanuki}]{hyperref}
\usepackage{graphicx}
\usepackage{pdfpages}
\usepackage{listings}
\usepackage{lscape}
\allsectionsfont{\rmfamily}
\usepackage{fancybox}
\setcounter{secnumdepth}{4}
\usepackage{latexsym}
\usepackage{pifont}
\renewcommand\FrenchLabelItem{\textbullet}
\usepackage{setspace}
\widowpenalty=9999
\clubpenalty=9999
\parskip=6pt
\renewcommand{\rmdefault}{ptm} 




\begin{document}

\setlength{\parindent}{0pt}
\renewcommand\labelitemii{\ding{220}}
\begin{titlepage}
\includepdf[pages={1}]{page_de_garde_droit_fiscal.pdf}
\end{titlepage}

\newpage

\begin{center}

\vspace{0.5cm} 

\vspace{0.5cm}


\end{center}
\vspace{6cm}

\LARGE
\begin{center}
\LARGE

\textbf{\textsc{Version du \today}}\\

\end{center}

\vspace{12.0cm}

\normalsize

	




%\end{titlepage}





\fancyhf{} \fancyfoot[LE,RO]{\bfseries\thepage}

\renewcommand{\headrulewidth}{0.5pt}
\addtolength{\headheight}{0.5pt}
\renewcommand{\footrulewidth}{0pt}
\fancypagestyle{plain}{ \fancyhead{}
\renewcommand{\headrulewidth}{0pt}}

\thispagestyle{empty}
\setcounter{page}{0}
\null

\pagestyle{plain}
 


\thispagestyle{empty}
\setcounter{page}{0}
\null
\thispagestyle{empty}
\chapter*{Remerciements} 
\thispagestyle{empty}
\pagenumbering{roman} \setcounter{page}{1} 


Hélène Thomas a dit~: «~\textit{Le monde devrait remercier cette étonnante cohorte de gens qui font toujours preuve d'une insolente et illogique gentillesse.}~» \\

N'étant pas le monde mais ayant reçu une aide et des conseils avisés de certaines personnes, il me semblerait inadéquat de ne pas les remercier pour leur apport.\\

S'il est bien une personne que je dois remercier, c'est mon épouse Nathalie qui a supporté mes absences et mon travail pour me permettre d’obtenir ce master complémentaire. \\

Je tiens pour finir à remercier Madame Anne-Marie WOLLSEIFEN pour sa relecture.

\thispagestyle{empty}
 \newpage

\thispagestyle{empty}
\setcounter{page}{0}
\null
 \newpage



\addcontentsline{toc}{chapter}{Remerciements} 

\tableofcontents

\thispagestyle{empty}
 \newpage
\setcounter{page}{0}
\null
\thispagestyle{empty}
 

\chapter{Introduction}

\pagestyle{plain}
\pagenumbering{arabic} \setcounter{page}{1} 

Pour rappel, la fraude fiscale peut être définie comme une « violation de la loi fiscale, qui comporte généralement une altération de la vérité (élément matériel), et qui est commise dans le but d’éviter ou de diminuer une charge fiscale (élément intentionnel) » \footnote{J. Kirkpatrick et D. Garabedian, Le régime fiscal des sociétés en Belgique, 3e éd., Bruxelles, Bruylant, 2003, p. 55, no 1.27.}. L’« altération de la vérité à l’égard du fisc » peut ainsi se situer soit au stade des déclarations adressées au fisc (fraude fiscale « ordinaire », incriminée à l’article 449 du CIR92), soit en amont, au stade des comptes annuels que les contribuables doivent tenir ou des actes juridiques qu’ils concluent entre eux et qui déterminent la consistance de leur matière/base imposable (fraude fiscale « complexe », « aggravée » par un faux en écriture, incriminée à l’article 450 du CIR92).

Le mensonge mis en forme dans un acte juridique est ce que le droit des obligations appelle une « simulation » : les parties « font un acte apparent dont elles conviennent de modifier ou de détruire les effets par une autre convention, demeurée secrète »\footnote{H. De Page, Traité élémentaire de droit civil belge, t. II, Bruxelles, Bruylant, 1964, no 618.}.

La Cour de cassation considère qu’il n’y a pas de « simulation prohibée à l’égard du fisc », mais qu’il y a « évitement licite de l’impôt » (ou « libre choix de la voie la moins imposée »), lorsque des contribuables concluent entre eux des actes juridiques dans le but d’éviter ou de diminuer leur charge fiscale, tant qu’ils en acceptent toutes les conséquences juridiques, quand bien même ces actes sembleraient, aux yeux du fisc, anormaux (rejet de la théorie de la « fraude à la loi » \footnote{Cass., 6 juin 1961, Brepols, Pas., I, p. 1082, J.D.F., 1961, p. 274.}), voire dénués de toute substance ou rationalité économique ou justifiés par des motifs exclusivement fiscaux (rejet de la théorie de la « réalité économique » \footnote{Cass., 22 mars 1990, Au Vieux Saint-Martin, Pas., I, p. 849, J.D.F., 1990, p. 110 ; Cass., 27 février 1987, Pas., I, p. 777, R.G.F., 1987, p. 183, J.D.F., 1988, p. 332 ; Cass., 29 janvier 1988, Pas., I, p. 633, conclusions du procureur général E. Krings, J.D.F., 1988, p. 338}.

En d’autres termes : « du moment que les parties établissent des actes dont elles acceptent toutes les conséquences, il n’y a pas simulation, et partant pas de fraude fiscale, même si le procédé utilisé a pour but de faire bénéficier d’un régime fiscal plus favorable et n’est pas le plus  normal. En procédant de la sorte, les parties ne font qu’user de la liberté des conventions et ne violent aucune obligation légale »\footnote{KIRKPATRICK, Le régime fiscal des sociétés en Belgique, Bruylant, 1992, p.21.}.

L’intention poursuivie par les parties est donc sans importance, quand bien même elle viserait exclusivement ou principalement à éviter ou à réduire une imposition. Par ailleurs, lorsqu’un même résultat économique peut être obtenu en recourant à plusieurs procédés juridiques, la liberté des conventions permet de choisir celle qui est la plus favorable du point de vue fiscal, sans qu’il y ait à rechercher la voie la plus normale.

Voici donc le cadre général posé depuis des décennies à tous les fiscalistes dont la mission est de conseiller au mieux leurs clients dans leur démarche afin de monter des projets légaux afin d'échapper à une taxation peu favorable.

Le législateur a donc cherché à définir et redéfinir cette notion d'abus fiscal en la gravant dans son marbre à savoir les codes fiscaux. Différents codes, différentes législation, ... que pourra donner leur confrontation ?

\chapter{La notion d'abus fiscal}

\section{Des principes généraux}
\subsection{Abus d’un droit en vue de créer un dommage}
Le principe de l’abus de droit se retrouve déjà en droit romain où, après une longue évolution, il est admis que celui qui énuclée l’œil de son attaquant pour se défendre n’encourt aucune sanction, sauf si telle était son unique intention. L’exercice d’un droit dans le seul but de causer un dommage est illicite \footnote{G. LONDERS, S. MOSSELMANS et A. BOSSUYT, «Deux principes  généraux du droit issus du droit national et du droit communautaire : l’enrichissement sans cause ou l’enrichissement injustifié et l’interdiction de l’abus de droit», in Cinquantième anniversaire des Traités de Rome – Cour de justice des Communautés européennes, Luxembourg, le 26 mars 2007.}. La notion « d’abus de droit » est commune à la plupart des Codes civils \footnote{J. MALHERBE, «L’abus de droit en droit fiscal comparé», rapport au colloque «Abuso di diritto», Université Luiss, Rome,juillet 2008.}.

Ce principe de droit veut sanctionner une manière d’agir qui excède un exercice normalement diligent du droit (ce qui implique une approche plutôt objective), ou une manière d’agir qui n’est pas conforme à un exercice loyal du droit (ce qui implique plutôt une approche subjective) \footnote{G. LONDERS, S. MOSSELMANS et A. BOSSUYT, op. cit., p. 111.}.

Selon la Cour de cassation belge, l’abus de droit peut résulter non seulement de l’exercice d’un droit avec la seule intention de nuire, mais aussi de l’exercice de ce droit d’une manière qui dépasse les limites de l’exercice normal de celui-ci par une personne normale et diligente. Il s’agit d’un principe général de droit consacré notamment par les articles 1382 et 1383 du Code civil \footnote{Cass., 10 septembre 1971, Revue critique jurisprudence belge, p. 300 avec note P. VAN OMMESLAGHE.}. Dans un arrêt du 31 octobre 2003, la Cour de cassation belge a considéré « qu’une procédure (judiciaire) peut revêtir un caractère vexatoire non seulement lorsqu’une partie est animée de l’intention de nuire mais aussi lorsqu’elle exerce son droit d’agir en justice d’une manière qui excède manifestement les limites de l’exercice normal de ce droit par une personne prudente et diligente » \footnote{Cass., 31 octobre 2003, J.T., 2004, p. 134 et les observations J.-Fr. VAN DROOGENBROECK, «L’abus procédural : une étape décisive», pp. 135 et s.; X. TATON, «Les irrégularités, nullités et abus de procédure», in Le procès civil accéléré? (J. ENGLEBERT), éd. Larcier, 2008 p. 233.}.

En droit fiscal belge, cette problématique est d’autant plus délicate que l’administration dispose du privilège du préalable.

\subsection{Simulation en vue de contourner une interdiction ou d’obtenir un avantage}
La notion d’abus de droit est fréquemment utilisée pour sanctionner des cas de simulation qui consistent à ne pas accepter les conséquences d’un acte juridique apparent ou à tromper les autorités en vue d’obtenir un avantage. Tel est par exemple le cas des mariages blancs, ou l’interdiction du mariage d’un médecin avec un de ses patients en vue de contourner l’interdiction de donations ou de legs\footnote{B. AUDIT, La fraude à la loi, Dalloz, 1974, p. 3.}.

\subsection{Simulation et fraude à la loi}
La simulation consiste à ne pas accepter toutes les conséquences juridiques de l’acte apparent, par exemple, en vue de tromper le fisc.

La « fraude à la loi » est l’utilisation d’un procédé juridique non simulé et licite en lui-même, mais anormal en vue d’échapper à une loi impérative ou d’ordre public qui autrement aurait été applicable. Cela viserait plutôt ce qu’on appelle l’ingénierie fiscale.

L’article 17 de la Convention de sauvegarde des droits de l’homme et des libertés fondamentales du 4 novembre 1950 prévoit sous le titre « Interdiction de l’abus de droit » qu’aucune des dispositions de la Convention ne peut être interprétée comme impliquant pour un État, un groupement ou un individu, un droit quelconque de se livrer à une activité ou d’accomplir un acte visant à la destruction de droits ou libertés reconnus dans la Convention ou à des limitations plus amples de ces droits et libertés que celles prévues à ladite Convention. C’est le principe que l’on ne peut utiliser les droits de l’homme en vue de détruire ces mêmes droits de l’homme.

En effet, le droit fiscal impose des obligations et n’accorde pas des droits au même titre que le droit de propriété ou le droit à la liberté d’expression, même si, comme on le verra plus loin, on pourrait être tenté de qualifier de droits le remboursement de prélèvements agricoles à l’exportation ou le droit à déduction en T.V.A.

\section{Abus fiscal en T.V.A.}
\section{Abus fiscal dans le CIR92}
\section{Abus fiscal dans le Code des droits d'enregistrement}

\chapter{Confrontation}
\chapter{Application au démembrement du droit de propriété}
\chapter{Conclusions}

\newpage


\null
\thispagestyle{empty}

\newpage

\null
\thispagestyle{empty}



\nocite{*}
\bibliographystyle{plain}
\bibliography{publications2}       % 'publications' is the name of a BibTeX file
\addcontentsline{toc}{chapter}{Bibliographie} 
\end{document}