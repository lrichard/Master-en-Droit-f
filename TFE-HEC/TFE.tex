\documentclass[12pt]{book}
\usepackage[utf8]{inputenc}
\usepackage[T1]{fontenc}
\usepackage{lmodern}
\usepackage[a4paper,left=3.5cm,right=1.5cm,top=3.0cm,bottom=2.5cm]{geometry}
\usepackage[frenchb]{babel}
\usepackage{fncychap}
\usepackage{fancyhdr}
\usepackage{sectsty}
\usepackage[pdftitle={Titre du TFE},pdfauthor={Laurent RICHARD}, pdfsubject={Travail de fin d'études - Master complémentaire en Droit fiscal}, pdfkeywords={Laurent, RICHARD, TVA, Tanuki}]{hyperref}
\usepackage{graphicx}
\usepackage{pdfpages}
\usepackage{listings}
\usepackage{lscape}
\allsectionsfont{\rmfamily}
\usepackage{fancybox}
\setcounter{secnumdepth}{4}
\usepackage{latexsym}
\usepackage{pifont}
\renewcommand\FrenchLabelItem{\textbullet}
\usepackage{setspace}
\widowpenalty=9999
\clubpenalty=9999
\parskip=6pt




\begin{document}

\setlength{\parindent}{0pt}
\renewcommand\labelitemii{\ding{220}}
\begin{titlepage}

\begin{center}

\vspace{0.5cm} 

\vspace{0.5cm}


\end{center}
\vspace{6cm}

\LARGE
\begin{center}
\LARGE
\textbf{\textsc{TITRE}}\\
\normalsize
\textbf{\textsc{Version du \today}}\\

\end{center}

\vspace{12.0cm}

\normalsize

\begin{minipage}{0.60\linewidth}
Promoteur :\\
Prénom NOM\\
Lecteur :\\
Prénom NOM\\	
\end{minipage}
\begin{minipage}{0.45\linewidth}

\flushleft{
Travail de fin d'études présenté par\\
\textbf{Laurent RICHARD}\\
En vue de l'obtention du diplôme de\\
Master complémentaire en Droit fiscal\\ 
Option Fiscalité des entreprises\\
Année académique 2012-2013}
\end{minipage}



\end{titlepage}

\newpage
\onehalfspacing
\thispagestyle{empty}
\setcounter{page}{0}
\null

\newpage

\thispagestyle{empty}
\setcounter{page}{0}
\null



\fancyhf{} \fancyfoot[LE,RO]{\bfseries\thepage}

\renewcommand{\headrulewidth}{0.5pt}
\addtolength{\headheight}{0.5pt}
\renewcommand{\footrulewidth}{0pt}
\fancypagestyle{plain}{ \fancyhead{}
\renewcommand{\headrulewidth}{0pt}}

\thispagestyle{empty}
\setcounter{page}{0}
\null

\pagestyle{plain}
 
 \newpage

\thispagestyle{empty}
\setcounter{page}{0}
\null
\thispagestyle{empty}
\chapter*{Remerciements} 
\thispagestyle{empty}
\pagenumbering{roman} \setcounter{page}{1} 

Hélène Thomas a dit~: «~\textit{Le monde devrait remercier cette étonnante cohorte de gens qui font toujours preuve d'une insolente et illogique gentillesse.}~» \\

N'étant pas le monde mais ayant reçu une aide et des conseils avisés de certaines personnes, il me semblerait inadéquat de ne pas les remercier pour leur apport.\\

S'il est bien une personne que je dois remercier, c'est mon épouse Nathalie qui a supporté mes absences et mon travail pour me permettre d’obtenir ce certificat. \\

La deuxième personne est Monsieur Alain Quewet, CISO du SPF Santé Publique. Monsieur Quewet a accepté de me coacher sur ce long chemin qu'est la mise en place d'un tel projet à l'AFMPS.\\ 

Je souhaite également remercier Monsieur Daniel Letecheur du FedICT pour son analyse et le temps passé avec son expérience et ses conseils pour mettre en place la méthode Quick-Win concernant l'analyse des risques ICT.\\

Je tiens pour finir à remercier Monsieur Axel Krausch-Lacroix pour sa relecture.

\thispagestyle{empty}
 \newpage

\thispagestyle{empty}
\setcounter{page}{0}
\null
 \newpage

\thispagestyle{empty}
\setcounter{page}{0}
\null
 \newpage
 
\thispagestyle{empty}
\setcounter{page}{0}
\null

\addcontentsline{toc}{chapter}{Remerciements} 

\tableofcontents

\thispagestyle{empty}
 \newpage
\setcounter{page}{0}
\null
\thispagestyle{empty}
 


\chapter{Introduction}

\pagestyle{plain}
\pagenumbering{arabic} \setcounter{page}{1} 

Eodem tempore etiam Hymetii praeclarae indolis viri negotium est actitatum, cuius hunc novimus esse textum. cum Africam pro consule regeret Carthaginiensibus victus inopia iam lassatis, ex horreis Romano populo destinatis frumentum dedit, pauloque postea cum provenisset segetum copia, integre sine ulla restituit mora.

Sed laeditur hic coetuum magnificus splendor levitate paucorum incondita, ubi nati sunt non reputantium, sed tamquam indulta licentia vitiis ad errores lapsorum ac lasciviam. ut enim Simonides lyricus docet, beate perfecta ratione vieturo ante alia patriam esse convenit gloriosam.

\section{Prout}
Pandente itaque viam fatorum sorte tristissima, qua praestitutum erat eum vita et imperio spoliari, itineribus interiectis permutatione iumentorum emensis venit Petobionem oppidum Noricorum, ubi reseratae sunt insidiarum latebrae omnes, et Barbatio repente apparuit comes, qui sub eo domesticis praefuit, cum Apodemio agente in rebus milites ducens, quos beneficiis suis oppigneratos elegerat imperator certus nec praemiis nec miseratione ulla posse deflecti.

Non ergo erunt homines deliciis diffluentes audiendi, si quando de amicitia, quam nec usu nec ratione habent cognitam, disputabunt. Nam quis est, pro deorum fidem atque hominum! qui velit, ut neque diligat quemquam nec ipse ab ullo diligatur, circumfluere omnibus copiis atque in omnium rerum abundantia vivere? Haec enim est tyrannorum vita nimirum, in qua nulla fides, nulla caritas, nulla stabilis benevolentiae potest esse fiducia, omnia semper suspecta atque sollicita, nullus locus amicitiae.

\subsection{Re-prout}
Has autem provincias, quas Orontes ambiens amnis imosque pedes Cassii montis illius celsi praetermeans funditur in Parthenium mare, Gnaeus Pompeius superato Tigrane regnis Armeniorum abstractas dicioni Romanae coniunxit.

Sin autem ad adulescentiam perduxissent, dirimi tamen interdum contentione vel uxoriae condicionis vel commodi alicuius, quod idem adipisci uterque non posset. Quod si qui longius in amicitia provecti essent, tamen saepe labefactari, si in honoris contentionem incidissent; pestem enim nullam maiorem esse amicitiis quam in plerisque pecuniae cupiditatem, in optimis quibusque honoris certamen et gloriae; ex quo inimicitias maximas saepe inter amicissimos exstitisse.

\section{Haaaaa}
Haec dum oriens diu perferret, caeli reserato tepore Constantius consulatu suo septies et Caesaris ter egressus Arelate Valentiam petit, in Gundomadum et Vadomarium fratres Alamannorum reges arma moturus, quorum crebris excursibus vastabantur confines limitibus terrae Gallorum.

Iamque lituis cladium concrepantibus internarum non celate ut antea turbidum saeviebat ingenium a veri consideratione detortum et nullo inpositorum vel conpositorum fidem sollemniter inquirente nec discernente a societate noxiorum insontes velut exturbatum e iudiciis fas omne discessit, et causarum legitima silente defensione carnifex rapinarum sequester et obductio capitum et bonorum ubique multatio versabatur per orientales provincias, quas recensere puto nunc oportunum absque Mesopotamia digesta, cum bella Parthica dicerentur, et Aegypto, quam necessario aliud reieci ad tempus\footnote{Haec igitur prima lex amicitiae sanciatur, ut ab amicis honesta petamus, amicorum causa honesta faciamus, ne exspectemus quidem, dum rogemur; studium semper adsit, cunctatio absit; consilium vero dare audeamus libere. Plurimum in amicitia amicorum bene suadentium valeat auctoritas, eaque et adhibeatur ad monendum non modo aperte sed etiam acriter, si res postulabit, et adhibitae pareatur}.

Illud autem non dubitatur quod cum esset aliquando virtutum omnium domicilium Roma, ingenuos advenas plerique nobilium, ut Homerici bacarum suavitate Lotophagi, humanitatis multiformibus officiis retentabant.

Post hoc impie perpetratum quod in aliis quoque iam timebatur, tamquam licentia crudelitati indulta per suspicionum nebulas aestimati quidam noxii damnabantur. quorum pars necati, alii puniti bonorum multatione actique laribus suis extorres nullo sibi relicto praeter querelas et lacrimas, stipe conlaticia victitabant, et civili iustoque imperio ad voluntatem converso cruentam, claudebantur opulentae domus et clarae.

Alii nullo quaerente vultus severitate adsimulata patrimonia sua in inmensum extollunt, cultorum ut puta feracium multiplicantes annuos fructus, quae a primo ad ultimum solem se abunde iactitant possidere, ignorantes profecto maiores suos, per quos ita magnitudo Romana porrigitur, non divitiis eluxisse sed per bella saevissima, nec opibus nec victu nec indumentorum vilitate gregariis militibus discrepantes opposita cuncta superasse virtute.

Mox dicta finierat, multitudo omnis ad, quae imperator voluit, promptior laudato consilio consensit in pacem ea ratione maxime percita, quod norat expeditionibus crebris fortunam eius in malis tantum civilibus vigilasse, cum autem bella moverentur externa, accidisse plerumque luctuosa, icto post haec foedere gentium ritu perfectaque sollemnitate imperator Mediolanum ad hiberna discessit.

Quare talis improborum consensio non modo excusatione amicitiae tegenda non est sed potius supplicio omni vindicanda est, ut ne quis concessum putet amicum vel bellum patriae inferentem sequi; quod quidem, ut res ire coepit, haud scio an aliquando futurum sit. Mihi autem non minori curae est, qualis res publica post mortem meam futura, quam qualis hodie sit.

Hinc ille commotus ut iniusta perferens et indigna praefecti custodiam protectoribus mandaverat fidis. quo conperto Montius tunc quaestor acer quidem sed ad lenitatem propensior, consulens in commune advocatos palatinarum primos scholarum adlocutus est mollius docens nec decere haec fieri nec prodesse addensque vocis obiurgatorio sonu quod si id placeret, post statuas Constantii deiectas super adimenda vita praefecto conveniet securius cogitari.

Superatis Tauri montis verticibus qui ad solis ortum sublimius attolluntur, Cilicia spatiis porrigitur late distentis dives bonis omnibus terra, eiusque lateri dextro adnexa Isauria, pari sorte uberi palmite viget et frugibus minutis, quam mediam navigabile flumen Calycadnus interscindit.

Itaque tum Scaevola cum in eam ipsam mentionem incidisset, exposuit nobis sermonem Laeli de amicitia habitum ab illo secum et cum altero genero, C. Fannio Marci filio, paucis diebus post mortem Africani. Eius disputationis sententias memoriae mandavi, quas hoc libro exposui arbitratu meo; quasi enim ipsos induxi loquentes, ne 'inquam' et 'inquit' saepius interponeretur, atque ut tamquam a praesentibus coram haberi sermo videretur.

Orientis vero limes in longum protentus et rectum ab Euphratis fluminis ripis ad usque supercilia porrigitur Nili, laeva Saracenis conterminans gentibus, dextra pelagi fragoribus patens, quam plagam Nicator Seleucus occupatam auxit magnum in modum, cum post Alexandri Macedonis obitum successorio iure teneret regna Persidis, efficaciae inpetrabilis rex, ut indicat cognomentum.

Ergo ego senator inimicus, si ita vultis, homini, amicus esse, sicut semper fui, rei publicae debeo. Quid? si ipsas inimicitias, depono rei publicae causa, quis me tandem iure reprehendet, praesertim cum ego omnium meorum consiliorum atque factorum exempla semper ex summorum hominum consiliis atque factis mihi censuerim petenda.

Restabat ut Caesar post haec properaret accitus et abstergendae causa suspicionis sororem suam, eius uxorem, Constantius ad se tandem desideratam venire multis fictisque blanditiis hortabatur. quae licet ambigeret metuens saepe cruentum, spe tamen quod eum lenire poterit ut germanum profecta, cum Bithyniam introisset, in statione quae Caenos Gallicanos appellatur, absumpta est vi febrium repentina. cuius post obitum maritus contemplans cecidisse fiduciam qua se fultum existimabat, anxia cogitatione, quid moliretur haerebat.


\chapter{Introduction}




Eodem tempore etiam Hymetii praeclarae indolis viri negotium est actitatum, cuius hunc novimus esse textum. cum Africam pro consule regeret Carthaginiensibus victus inopia iam lassatis, ex horreis Romano populo destinatis frumentum dedit, pauloque postea cum provenisset segetum copia, integre sine ulla restituit mora.

Sed laeditur hic coetuum magnificus splendor levitate paucorum incondita, ubi nati sunt non reputantium, sed tamquam indulta licentia vitiis ad errores lapsorum ac lasciviam. ut enim Simonides lyricus docet, beate perfecta ratione vieturo ante alia patriam esse convenit gloriosam.

Pandente itaque viam fatorum sorte tristissima, qua praestitutum erat eum vita et imperio spoliari, itineribus interiectis permutatione iumentorum emensis venit Petobionem oppidum Noricorum, ubi reseratae sunt insidiarum latebrae omnes, et Barbatio repente apparuit comes, qui sub eo domesticis praefuit, cum Apodemio agente in rebus milites ducens, quos beneficiis suis oppigneratos elegerat imperator certus nec praemiis nec miseratione ulla posse deflecti.

Non ergo erunt homines deliciis diffluentes audiendi, si quando de amicitia, quam nec usu nec ratione habent cognitam, disputabunt. Nam quis est, pro deorum fidem atque hominum! qui velit, ut neque diligat quemquam nec ipse ab ullo diligatur, circumfluere omnibus copiis atque in omnium rerum abundantia vivere? Haec enim est tyrannorum vita nimirum, in qua nulla fides, nulla caritas, nulla stabilis benevolentiae potest esse fiducia, omnia semper suspecta atque sollicita, nullus locus amicitiae.

Has autem provincias, quas Orontes ambiens amnis imosque pedes Cassii montis illius celsi praetermeans funditur in Parthenium mare, Gnaeus Pompeius superato Tigrane regnis Armeniorum abstractas dicioni Romanae coniunxit.

Sin autem ad adulescentiam perduxissent, dirimi tamen interdum contentione vel uxoriae condicionis vel commodi alicuius, quod idem adipisci uterque non posset. Quod si qui longius in amicitia provecti essent, tamen saepe labefactari, si in honoris contentionem incidissent; pestem enim nullam maiorem esse amicitiis quam in plerisque pecuniae cupiditatem, in optimis quibusque honoris certamen et gloriae; ex quo inimicitias maximas saepe inter amicissimos exstitisse.

Haec dum oriens diu perferret, caeli reserato tepore Constantius consulatu suo septies et Caesaris ter egressus Arelate Valentiam petit, in Gundomadum et Vadomarium fratres Alamannorum reges arma moturus, quorum crebris excursibus vastabantur confines limitibus terrae Gallorum.

Iamque lituis cladium concrepantibus internarum non celate ut antea turbidum saeviebat ingenium a veri consideratione detortum et nullo inpositorum vel conpositorum fidem sollemniter inquirente nec discernente a societate noxiorum insontes velut exturbatum e iudiciis fas omne discessit, et causarum legitima silente defensione carnifex rapinarum sequester et obductio capitum et bonorum ubique multatio versabatur per orientales provincias, quas recensere puto nunc oportunum absque Mesopotamia digesta, cum bella Parthica dicerentur, et Aegypto, quam necessario aliud reieci ad tempus.

Haec igitur prima lex amicitiae sanciatur, ut ab amicis honesta petamus, amicorum causa honesta faciamus, ne exspectemus quidem, dum rogemur; studium semper adsit, cunctatio absit; consilium vero dare audeamus libere. Plurimum in amicitia amicorum bene suadentium valeat auctoritas, eaque et adhibeatur ad monendum non modo aperte sed etiam acriter, si res postulabit, et adhibitae pareatur.

Illud autem non dubitatur quod cum esset aliquando virtutum omnium domicilium Roma, ingenuos advenas plerique nobilium, ut Homerici bacarum suavitate Lotophagi, humanitatis multiformibus officiis retentabant.

Post hoc impie perpetratum quod in aliis quoque iam timebatur, tamquam licentia crudelitati indulta per suspicionum nebulas aestimati quidam noxii damnabantur. quorum pars necati, alii puniti bonorum multatione actique laribus suis extorres nullo sibi relicto praeter querelas et lacrimas, stipe conlaticia victitabant, et civili iustoque imperio ad voluntatem converso cruentam, claudebantur opulentae domus et clarae.

Alii nullo quaerente vultus severitate adsimulata patrimonia sua in inmensum extollunt, cultorum ut puta feracium multiplicantes annuos fructus, quae a primo ad ultimum solem se abunde iactitant possidere, ignorantes profecto maiores suos, per quos ita magnitudo Romana porrigitur, non divitiis eluxisse sed per bella saevissima, nec opibus nec victu nec indumentorum vilitate gregariis militibus discrepantes opposita cuncta superasse virtute.

Mox dicta finierat, multitudo omnis ad, quae imperator voluit, promptior laudato consilio consensit in pacem ea ratione maxime percita, quod norat expeditionibus crebris fortunam eius in malis tantum civilibus vigilasse, cum autem bella moverentur externa, accidisse plerumque luctuosa, icto post haec foedere gentium ritu perfectaque sollemnitate imperator Mediolanum ad hiberna discessit.

Quare talis improborum consensio non modo excusatione amicitiae tegenda non est sed potius supplicio omni vindicanda est, ut ne quis concessum putet amicum vel bellum patriae inferentem sequi; quod quidem, ut res ire coepit, haud scio an aliquando futurum sit. Mihi autem non minori curae est, qualis res publica post mortem meam futura, quam qualis hodie sit.

Hinc ille commotus ut iniusta perferens et indigna praefecti custodiam protectoribus mandaverat fidis. quo conperto Montius tunc quaestor acer quidem sed ad lenitatem propensior, consulens in commune advocatos palatinarum primos scholarum adlocutus est mollius docens nec decere haec fieri nec prodesse addensque vocis obiurgatorio sonu quod si id placeret, post statuas Constantii deiectas super adimenda vita praefecto conveniet securius cogitari.

Superatis Tauri montis verticibus qui ad solis ortum sublimius attolluntur, Cilicia spatiis porrigitur late distentis dives bonis omnibus terra, eiusque lateri dextro adnexa Isauria, pari sorte uberi palmite viget et frugibus minutis, quam mediam navigabile flumen Calycadnus interscindit.

Itaque tum Scaevola cum in eam ipsam mentionem incidisset, exposuit nobis sermonem Laeli de amicitia habitum ab illo secum et cum altero genero, C. Fannio Marci filio, paucis diebus post mortem Africani. Eius disputationis sententias memoriae mandavi, quas hoc libro exposui arbitratu meo; quasi enim ipsos induxi loquentes, ne 'inquam' et 'inquit' saepius interponeretur, atque ut tamquam a praesentibus coram haberi sermo videretur.

Orientis vero limes in longum protentus et rectum ab Euphratis fluminis ripis ad usque supercilia porrigitur Nili, laeva Saracenis conterminans gentibus, dextra pelagi fragoribus patens, quam plagam Nicator Seleucus occupatam auxit magnum in modum, cum post Alexandri Macedonis obitum successorio iure teneret regna Persidis, efficaciae inpetrabilis rex, ut indicat cognomentum.

Ergo ego senator inimicus, si ita vultis, homini, amicus esse, sicut semper fui, rei publicae debeo. Quid? si ipsas inimicitias, depono rei publicae causa, quis me tandem iure reprehendet, praesertim cum ego omnium meorum consiliorum atque factorum exempla semper ex summorum hominum consiliis atque factis mihi censuerim petenda.

Restabat ut Caesar post haec properaret accitus et abstergendae causa suspicionis sororem suam, eius uxorem, Constantius ad se tandem desideratam venire multis fictisque blanditiis hortabatur. quae licet ambigeret metuens saepe cruentum, spe tamen quod eum lenire poterit ut germanum profecta, cum Bithyniam introisset, in statione quae Caenos Gallicanos appellatur, absumpta est vi febrium repentina. cuius post obitum maritus contemplans cecidisse fiduciam qua se fultum existimabat, anxia cogitatione, quid moliretur haerebat.

\newpage


\null
\thispagestyle{empty}

\newpage

\null
\thispagestyle{empty}



\nocite{*}
\bibliographystyle{plain}
\bibliography{publications2}       % 'publications' is the name of a BibTeX file
\addcontentsline{toc}{chapter}{Bibliographie} 
\end{document}