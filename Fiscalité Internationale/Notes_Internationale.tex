\documentclass{book}
\usepackage[utf8]{inputenc}
\usepackage[T1]{fontenc}
\usepackage{lmodern}
\usepackage[a4paper,left=3.5cm,right=2.5cm,top=2.5cm,bottom=2cm]{geometry}
\usepackage[frenchb]{babel}
\usepackage{fncychap}
\usepackage{fancyhdr}
\usepackage{sectsty}
\usepackage[pdftitle={Master complémentaire en droit fiscal - Eléments de fiscalité locale et régionale},pdfauthor={Laurent RICHARD}, pdfsubject={Notes de cours}, pdfkeywords={ULg, Fisc}]{hyperref}
\usepackage{graphicx}
\usepackage{pdfpages}
\usepackage{listings}
\usepackage{lscape}
\allsectionsfont{\sffamily}
\usepackage{fancybox}
\setcounter{secnumdepth}{4}
\usepackage{latexsym}
\usepackage{pifont}
\renewcommand\FrenchLabelItem{\textbullet}
\widowpenalty=9999
\clubpenalty=9999




\begin{document}
\sffamily
\newcommand{\RPoint}{\protect\includegraphics[height=1.7ex,keepaspectratio]{point.png}}
\newcommand{\RSave}{\protect\includegraphics[height=1.7ex,keepaspectratio]{Save.png}}
\renewcommand\labelitemii{\ding{220}}
\begin{titlepage}

\begin{center}
\begin{Large}ULg - Master complémentaire en droit fiscal\end{Large}\\
\vspace{0.5cm}- \\
\vspace{0.5cm}
Fiscalité des entreprises

\end{center}
\vspace{6cm}

\LARGE
\begin{center}
\textsc{Notes de cours - Eléments de fiscalité internationale}\\
\end{center}

\vspace{10.0cm}

\normalsize
\flushright{
\textbf{Laurent RICHARD}\\
Comptable-fiscaliste agréé IPCF\\
\vspace{0.5cm}
Année académique 2012-2013}

\end{titlepage}

\newpage

\thispagestyle{empty}
\setcounter{page}{0}
\null

\newpage
\thispagestyle{empty}
\setcounter{page}{0}
\vspace{20cm}

\vfill
\begin{flushright}
Life is a succession of lessons, \\
which must be lived to be understood. \\ 
--- \textit{Ralph Waldo Emerson}
\end{flushright}
\vfill
\newpage

\renewcommand{\chaptermark}[1]{\markboth{#1}{}}
\renewcommand{\sectionmark}[1]{\markright{\thesection\ #1}}
\fancyhf{} \fancyhead[LE,RO]{\bfseries\thepage}
\fancyhead[LO]{\bfseries\rightmark}
\fancyhead[RE]{\bfseries\leftmark}
\renewcommand{\headrulewidth}{0.5pt}
\addtolength{\headheight}{0.5pt}
\renewcommand{\footrulewidth}{0pt}
\fancypagestyle{plain}{ \fancyhead{}
\renewcommand{\headrulewidth}{0pt}}

\thispagestyle{empty}
\setcounter{page}{0}
\null
\newpage
\pagenumbering{roman} \setcounter{page}{1} 




\tableofcontents



\chapter{Cours du 04.10.2012}
\pagenumbering{arabic} \setcounter{page}{1} 

\section{Introduction}

Convention préventive de double imposition ... il existe un modèle OCDE.\\

Critère personne : Les USA taxent les américains sur base de leur nationalité.\\

Critère matériel : Exemple : les immeubles pour les non-résidents. L'établissement stable (et donc taxation de la société pour ses revenus générés en Belgique).\\

Risque de double imposition en cas d'AAB. \\

Ce qui n'est pas une double imposition juridique est une double imposition économique.\\

Droit de mutation par décès d'un immeuble (non-résident) : risque de double imposition \\

France : Revenu PAR MENAGE + Pouvoir d'imposition par convention -> Même si pas révu par une disposition française. Société : uniquement sur le revenu d'exploitation français sauf deux exceptions. \\


NL : Possibilité pour les non-résidents d'être taxé comme résident. -> Raison des déductions possibles. Imposition des sociétés sans personnalité juridique. Doelvermogens - Taxation du capital.\\

Notion de revenu net à la frontière pour les revenus étrangers.\\

Différence entre revenus cité mais non imposé (+ value) et revenus non coté. Point de vue de la Cour de Cassation. (Cfr Article 156 CIR - Arrêt Sidro - 15.09.1970) - Presomptaion d'imposition de 30\% des salaires bruts cfr Comm IR 153/23?\\

L'imputation permet de gommer l'imposition étranger pour avoir une charge globale équivalente à l'impot belge. Mais imputation avec un maximum du taux d'imposition de l'impot national. Capital export neutrality.\\

QFIE : Imputation de l'impot étranger avec un maximum de 15\% d'impot. Si impot 20\% sur 100 \\

QFIE : 80+ 14 (80*15/85) = 94 x 40\% Impot belge = 37,6 EUR - 14 (QFIE) -> 23,6 EUR à payer  \\

Si redevance 15\% même si taxation inférieure.\\

Pas de QFIE pour les dividendes.\\

Réduction de moitié (art 156 CIR).


\chapter{Cours du 18.10.2012}

Etudier avec le Code\\

Différence d'imposition entre location d'un bien en Belgique (sur base sur RC) ou location d'un bien à l'étranger (Loyer)


Article 4 Convention NL " Toutefois, cette expression ne comprend pas les personnes qui ne sont assujetties à l'impôt dans cet Etat que pour les revenus de sources situées dans cet Etat ou pour la fortune qui y est située." -> Pour exclure les résident nationaux pour des raisons matérielles.\\

Pour la convention, on distingue le pays de taxation mais pour chaque état, cela reste un résident (taxation à IPP/ISOC et pas INR-PP ou INR-SOC) pour les parties concédée par la Convention.\\

Attention : Leasing immobilier n'est pas un revenu immobilier en Belgue Art 10 §2 CIR\\

Art 156 : Imposés = Prévu (meme exempté) - Arret Sidro)\\
Dans la convention : Cfr circulaire - Différence entre imposé, imposable, réellement imposé.\\
Ici c'est imposé comme réellement imposé.\\

Valeur pas à 100\% de ciculaire mais on peut considéré que c'est la volonté des parties.\\

Système moniste VS système dualiste (nécessité de l'intégrer dans le droit national).\\

Attention, le Droit européen va primer sur les CPDI.\\

Réserver de progressivité : article 155 CIR. Pas de réserve de progressivité à l'ISOC.\\

Si taxation distincte avec exempte, l'Administration dit qu'il faut ajouté au revenu global obligatoirement.\\

Si nouvelles conventions, besoin de l'accord des régions/communautés. Les anciennes conventions s'appliquent quant même aux régions/communauté.\\

Intérets, amendes, cot. sec. sociales s'applique sur le montant hors exemption. Pas les majorations.\\

Traité de Vienne explique les interprétations/principe qu'il faut donner aux traités. Même aux personnes n'étant pas partie au traité.\\

Solution générale de résolution des conflits s'applique si problème 3, §2 et pas si mauvaise interprétation de la convention (Taxation dans l'Etat de Résidence dans ce cas). Si double non-imposition suite à une mauvaise interprétation, application hors convention -> Taxation ER.\\

Dans le cas de l'imputation, il n'y a pas de risque vu que l'exemption revoir une imputation de 0.




bart.peeters@hubrussel.be

La convention limite le droit national -> Vérifier si le droit national impose. S'il n'impose pas, il n'y a pas de raison d'aller voir la convention.

Article 5 §4 : uniquement si accessoire.

AAB entre société du groupe acceptable car but uniquement d'éluder l'impôt.


\nocite{*}
\bibliographystyle{plain}
\bibliography{publications2}       % 'publications' is the name of a BibTeX file
\addcontentsline{toc}{chapter}{Bibliographie} 
\end{document}