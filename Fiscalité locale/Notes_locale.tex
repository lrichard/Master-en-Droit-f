\documentclass{book}
\usepackage[utf8]{inputenc}
\usepackage[T1]{fontenc}
\usepackage{lmodern}
\usepackage[a4paper,left=3.5cm,right=2.5cm,top=2.5cm,bottom=2cm]{geometry}
\usepackage[frenchb]{babel}
\usepackage{fncychap}
\usepackage{fancyhdr}
\usepackage{sectsty}
\usepackage[pdftitle={Master complémentaire en droit fiscal - Eléments de fiscalité locale et régionale},pdfauthor={Laurent RICHARD}, pdfsubject={Notes de cours}, pdfkeywords={ULg, Fisc}]{hyperref}
\usepackage{graphicx}
\usepackage{pdfpages}
\usepackage{listings}
\usepackage{lscape}
\allsectionsfont{\sffamily}
\usepackage{fancybox}
\setcounter{secnumdepth}{4}
\usepackage{latexsym}
\usepackage{pifont}
\renewcommand\FrenchLabelItem{\textbullet}
\widowpenalty=9999
\clubpenalty=9999




\begin{document}
\sffamily
\newcommand{\RPoint}{\protect\includegraphics[height=1.7ex,keepaspectratio]{point.png}}
\newcommand{\RSave}{\protect\includegraphics[height=1.7ex,keepaspectratio]{Save.png}}
\renewcommand\labelitemii{\ding{220}}
\begin{titlepage}

\begin{center}
\begin{Large}ULg - Master complémentaire en droit fiscal\end{Large}\\
\vspace{0.5cm}- \\
\vspace{0.5cm}
Fiscalité des entreprises

\end{center}
\vspace{6cm}

\LARGE
\begin{center}
\textsc{Notes de cours - Eléments de fiscalité locale}\\
\end{center}

\vspace{10.0cm}

\normalsize
\flushright{
\textbf{Laurent RICHARD}\\
Comptable-fiscaliste agréé IPCF\\
\vspace{0.5cm}
Année académique 2012-2013}

\end{titlepage}

\newpage

\thispagestyle{empty}
\setcounter{page}{0}
\null

\newpage
\thispagestyle{empty}
\setcounter{page}{0}
\vspace{20cm}

\vfill
\begin{flushright}
Life is a succession of lessons, \\
which must be lived to be understood. \\ 
--- \textit{Ralph Waldo Emerson}
\end{flushright}
\vfill
\newpage

\renewcommand{\chaptermark}[1]{\markboth{#1}{}}
\renewcommand{\sectionmark}[1]{\markright{\thesection\ #1}}
\fancyhf{} \fancyhead[LE,RO]{\bfseries\thepage}
\fancyhead[LO]{\bfseries\rightmark}
\fancyhead[RE]{\bfseries\leftmark}
\renewcommand{\headrulewidth}{0.5pt}
\addtolength{\headheight}{0.5pt}
\renewcommand{\footrulewidth}{0pt}
\fancypagestyle{plain}{ \fancyhead{}
\renewcommand{\headrulewidth}{0pt}}

\thispagestyle{empty}
\setcounter{page}{0}
\null
\newpage
\pagenumbering{roman} \setcounter{page}{1} 




\tableofcontents



\chapter{Cours du 20.09.2012}
\pagenumbering{arabic} \setcounter{page}{1} 

\section{Introduction}

De plus en plus, la fiscalité locale va de nouveau éclore. Le rôle de secrétaire communal est très diversifié et  de plus en plus complexe (Jurisprudence, ....) -> Délégation de la défense des intérêts par des professionnels.\\

On ne peut séparer la fiscalité locale et régionale de son contexte politique et institutionnel.\\

Nous appliquons cela. Qui n'est que le résultat d'une négociation politique.\\

Examen : Cas pratique. Mise à disposition du décret, réglement communal. 2 ou 3 erreurs. Memo d'une page, une page et demi, deux pagew. Livre OUVERT. Comment relier le droit aux faits ?\\

Importance des pipe-lines du cours.\\

\section{Histoire} 

1780 : La Belgique est morcelée en duchés (Principauté de Liège, Duché de Brabant, Namur, Hainaut) reliés au Saint-Empire germanique. Tant qu'ils payent un tribut, ils ont une paix royale. Donc, popote interne. Importance importante en matière juridique. Le droit est coutumier mais pas de juge central. Fiscalité locale sans unité avec chaque ville faisant sa popote. Importance des chartes du Moyen-Age. Pas d'unité fiscale, économique  ou juridique.\\

Régime des corporations. Il faut etre membre pour exercer, pas de libre entreprise. Brigandage important. Beaucoup de gens dans les villes entourées de remparts. Toute l'économie dans les villes -> Taxation aisée.\\

Communication essentiellement par l'eau (transport de marchandises). Beaucoup de consommation locale\\

Socialement, régime de M-A. La fortune immobilière appartient à la noblesse. Peu de fortune mobilière. La noblesse est exemptée d'impots. Le clergé régulier est exempté d'impots. Les autres (les bourgeois ont une représentation dans le Tiers-Etat comme les maitres de corporation et les prestatires de services c'est à dire les avocats) vont lancer les révolutions de type intellectuels. Ils vont fédérer les énergies du peuple pour la révolution.\\

Clivage religieux et linguistique. On peut parler la langue que l'on veut mais la langue véhiculaire et usuel des élites est le français.\\

1789 : Révolution française. Les documents interdits et subversifs sont imprimées à Liège. Donc Révolution à Liège. Révolution de progressistes. Joseph II arrive à la Cour d'Autriche. Introduction de la peine pénale de confiscation des biens. Tous les autres états que Liège vont faire leur révolution contre Joseph II. La noblesse est contre le progrès et les libéraux (Tiers-Etat) pensent qu'il ne va pas assez vite. => Révolution de conservateur qui va réussir mais ne tenir que 10 mois avant le retour de la Révolution française. Toujours le meme système morcelé.\\

1795 : La France décide d'annexer la Belgique. Novembre. On fait une page blanche. Droit français s'applique. Plus de droit coutumier. 1804 : arrivée du Code civil. Idée de l'Etat fort, centralisé et unifié. Soumission de tout le monde au même droit. Fiscalement, même chose. Administrativement, les villes communes sont radiées, les municipalités arivent en France en 1863. Donc aucune orgage décisionnel/particularité locale autre que l'Etat. => Système fiscal commun à tous. Instauration des notaires. Fiscalité révolutionaire : égalitaire pour tous. Invention du droit de succession (droit de mutation par décès) au même tarif et sans exception. Invention du droit d'enregistrement. Mission aux notaires d'appliquer le Code civil. Fiscalité immobilière et patrimoniale. Invention des Contributions.\\

Contributions directes :

\begin{enumerate}
\item Contribution immobilière - Taxe fonçiere, 
\item Contribution personnel - Evaluation forfaitaire du revenu des gens. (Nombre de fenetres, nombre de chevaux domestiques, nombre de domestiques, nombre de cheminée) -> Taxation indiciaire.
\item Taxation de l'économie - Taxe de patente, impot des personnes physique de la catégorie revenus professionnel
\item Taxes de production (alcool, bière, sel, sucre) -> Droit d'accises.
\end{enumerate}

Tout revient à l'Etat, il y a des dotations et les additionnels pour les communautés locales (si le système central veut bien).\\

La noblesse et le clergé belge s'adaptent vite. Profil bas. Beaucoup de mariage entre la noblesse et des professions libérales. Nouvelle élite.\\

1790 : Décret d'Alarde. Liberté de commerce et d'industrie. Abolition des corporations. Libéralisation de l'activité économique. Essentiellement orienté vers les personnes physiques. La société civile n'a pas la personnalité juridique. Activité de groupe géré par la réglementation personnelle.\\

Logistiquement, pas beaucoup d'avancées. L'économie reste très locale. Linguistiquement, c'est le français partout. Les dialectes sont rayés. Plus d'accès à la Justice pour certains -> plafond de verre. 3\% de la population parle français. Religieusement, c'est catholique.\\

Parallélisme entre le système fiscal français et le belge. On peut comparer donc la France et la Belgique en 200 ans vu que le départ est le même. \\

Napoléon est un boulimique de travail mais rassemble l'information. Organisation de la Belgique. Jemappes, Bruxelles, Anvers, Gand, Hasselt, Arlon. Les 9 départements sont calculés sur base de 3 jours de diligence. La frontière linguistique actuelle suit le tracé napoléonien ...\\

Le droit de mutation (immobilière) se fait sur base du donneur/défunt.\\


1815 : Arrivée des Hollandais pour 15 ans. Devenu un Royaume de 1.5 million de personnes. La Belgique fait 3 millions d'habitants. On conserve le Code Napoléon (droit romain) mais les néerlandais sont plus axé sur le systèmes anglo-saxons. La société arrive (notion du système anglo-saxon). Le Code civil ne connait pas le patrimoine autre que le sien. Création de la société - personne morale en 1817. (France en 1862). Fiscalement, ce n'est pas dérangeant que les communes n'ont pas de droit. Apparition des provinces en tant qu'autorité politique -> fiscalité provinciale. Socialement, cela marche bien. \\

Le droit de mutation devient droit de succession (mutation mobilière et immobilière -> Taxation d'un patrimoine). La succession est basée sur les parts reçues par chaque héritier. Sauf le droit de mutation par décès pour l'immobilier pour les résidents étrangers. NB : La Flandres taxe de nouveau sur base du défunt pour calculer à partir de tante/nièce.\\

1825-1830, on arrive avec une nouvelle génération bouillonnante. L'économie va bien et il y a de l'argent. Clivage Catho-protestants. Les catholiques ont l'éducation et l'assistance publique. Donc, friction car les Hollandais veulent s'attaquer à l'enseignement. Lingusitiquement, on impose le néerlandais comme langue véhiculaire ... langue qui n'est connue par personne. Fiscalement, les hollandais veulent une réforme fiscale. Les hollandais étendent les contributions françaises. => Révolution.\\

On maintient le droit de succession (réforme en 1939), droit d'enregistrement (idem), contributions directes y compris la réforme fiscale hollandais, les droits d'accises (d'origine anglo-saxone).\\

Personne n'a retouché les faits générateurs depuis 1830 !!!\\

Quid du local ? Toujours dotations, impots partagé et additionnels. Tout est stable mais clivage religieux persistant (catholique VS non-catholiques ou libéraux) et linguistiques (retour du français comme SEULE langue en Belgique). 1840 : Création du mouvement flamand et de l'identité flamande (et non néerlandais) comme avant 1790. Mutation vers 1870 du néerlandais comme langue souhaitée. L'union fait la force (union de la noblesse et des professions libérales, des catholiques et des libéraux qui ont quand même combattus ensemble, ...) Point de vue administratif, on garde le Décret d'Allarde. Grand deal, on donne à tous ce qu'il veut. On met les titres de noblesse à titre honorifique. On laisse l'enseignement libre (pas d'enseignement d'Etat) pour les catholiques.\\

Fédéralisme (Fédéral décide sur le local) != Confédéralisme (le local delegue au fédéral). Pour le moment, tous les pouvoirs émanent de la nation => Fédéralisme car le confédéralisme brabansson n'a duré que 10 mois. Le pouvoir dépend des Chambres (Etat central). Mais en même temps, la Constitution prévoit des pouvoirs locaux, provinces et communes en matière d'intérets provincial et communal. Tout le monde est compétent pour tout. Il existe des balises gardant le caractère central de l'Etat belge (institutionnel et fiscale) :\\

\begin{itemize}
\item Art. 159 de la Constitution - Tous les tribuanux ne peuvent appliquer les réglements que pour autant qu'ils soient conforme à la Loi.-> Subsidiarité de la Loi. Les réglements sont subordonnées à la Loi.
\item L'Etat central peut intervenir dans les affaires communales et provinciales (mécanisme de la tutuelle). L'Etat peut se substituer. (Tutelle de substitution, Tutelle d'approbation préalable - condition suspensive et tutelle d'annulation - condition résolutoire)
\end{itemize}
\null
Constitution - Article 170 §3 et §4 (lever des impots sous reserve des limitations prévues par la Loi)

---

\begin{verbatim}
Art. 159

Les cours et tribunaux n'appliqueront les arrêtés et règlements généraux, provinciaux 
et locaux, qu'autant qu'ils seront conformes aux lois.


Art. 162

Les institutions provinciales et communales sont réglées par la loi.

La loi consacre l'application des principes suivants :
1 l'élection directe des membres des conseils provinciaux et communaux;
2 l'attribution aux conseils provinciaux et communaux de tout ce qui est d'intérêt 
provincial et communal, sans préjudice de l'approbation de leurs actes, dans les cas 
et suivant le mode que la loi détermine;
3 la décentralisation d'attributions vers les institutions provinciales et communales;
4 la publicité des séances des conseils provinciaux et communaux dans les limites établies 
par la loi;
5 la publicité des budgets et des comptes;
6 l'intervention de l'autorité de tutelle ou du pouvoir législatif fédéral, pour empêcher 
que la loi ne soit violée ou l'intérêt général blessé.

En exécution d'une loi adoptée à la majorité prévue à l'article 4, dernier alinéa, 
l'organisation  et l'exercice de la tutelle administrative peuvent être réglés par les 
Parlements de communauté ou de région.

En exécution d'une loi adoptée à la majorité prévue à l'article 4, dernier alinéa, le 
décret ou la règle visée à l'article 134 règle les conditions et le mode suivant lesquels 
plusieurs provinces ou plusieurs communes peuvent s'entendre ou s'associer. Toutefois, 
il ne peut être permis à plusieurs conseils provinciaux ou à plusieurs conseils communaux 
de délibérer en commun.

Art. 170

§ 1er. Aucun impôt au profit de l'État ne peut être établi que par une loi.

§ 2. Aucun impôt au profit de la communauté ou de la région ne peut être établi que 
par un décret ou une règle visée à l'article 134.

La loi détermine, relativement aux impositions visées à l'alinéa 1er, les exceptions 
dont la nécessité est démontrée.

§ 3. Aucune charge, aucune imposition ne peut être établie par la province que par une
décision de son conseil.

La loi détermine, relativement aux impositions visées à l'alinéa 1er, les exceptions dont
la nécessité est démontrée.

La loi peut supprimer en tout ou en partie les impositions visées à l'alinéa 1er.

§ 4. Aucune charge, aucune imposition ne peut être établie par l'agglomération, par la 
fédération de communes et par la commune que par une décision de leur conseil.

La loi détermine, relativement aux impositions visées à l'alinéa 1er, les exceptions dont 
la nécessité est démontrée.}

\end{verbatim}

\chapter{Cours du 25.09.2012}

Compétences Communes- Provinces\\

Le réglement sous tutelle d'approbation préalable nécessite une approbabtion préalable avant d'entrer en vigueur.\\

PIPELINE : En R.W, le processus pour les provinces et les communes aboutit au résultat suivant :


\section{Vote} 

(Acte législatif pour instauré un prélèvement fiscal) par le conseil provincial/communal -> Relire 170 de la Constitution "PAR" un acte du Conseil.\\

Article 1er du Protocole additionel à la CEDH établit un droit du cotoyen au respect de ces biens. On établit un droit fondamental à la propriété privée. L'Etat peut recouvrer les impôts qu'il établit. L'article 1er étbalit une règle générale. L'alinéa deux prévoit une exception (donc interprétation restrictive à donner à l'application des lois fiscales). Application de l'article 13 (toute dérogation au droit fondamental doit être appliquée sans discrimination) -> Légalité des impôts et l'égalité des impôts\\

D'où découle ... toute condition essentielle de l'impot doit être dans le texte voté au législatif (fixé par la Loi).\\

Jusqu'il y a peu, une dérogation du législatif à l'exécutif était illégal par rapport à la CEDH mais pas par rapport à la Constitution (cfr Article 159). La Cour constitutionnel permet de contrôler l'article 170 via l'extension de ces compétences 1989 (précipe d'égalité devant l'impôt).\\

Arrêt Engel (1976), la CEDH, vous pouvez faire une dérogation à condition : \\

\begin{itemize}
\item pertinent
\item objectif
\item motivé / justifié
\item proportionné
\end{itemize}
\null
Irruption du judiciaire pour juger du législation (Vandermersch - Procureur général auprès de la Cour de Cassation)\\

Arret Le Skie -> Tout peut etre contesté sur base des traités internationaux. -> extension du principe juridictionnel sur la vérification de l'égalité.\\

Le principe d'égalité n'est uniquement "ce n'est pas juste" :\\

\begin{itemize}
\item différence de traitement (mais situation identique)
\item traitements identiques (mais situation différente)
\end{itemize}
\null
Traitement identiques entre 2 personnes (taxe de 500 EUR pour un propriétaire de 400 m\up{2} et 3000 m\up{2}). Egalement traités mais pas dans la même situation. -> Principe d'égalité bafoué.\\

Si taxation au m\up{2}, plus dur d'attaquer. On doit vérifier si pertinent, objectif, justifié et proportionné.\\

Où trouver la justification : Exposé des motifs, commentaires de la loi (dans le texte), travaux préparatoires. Si pas de justification, pas moyen de controle -> AUTOMATIQUEMENT VIOLATION DE L'EGALITE mais on ne doit pas faire l'économie de mots. Il faut d'abord et avant d'arriver là, identifier une différence de traitement. Situer le débat (traitement identique ou différence de traitement) et ensuite dire que cette difference n'est pas justifié.\\

EN CE QUE .... ALORS QUE\\

Le conseiller communal doit avoir la justification au moment du vote. Pas de reconstitition à postériori. Justification votée.  \\

Pour qu'il y ait dans une différence de traiment et même situation, il y a violation du principe d'égalité qu'il soit MANIFESTEMENT disproportionné et déraisonnable.\\

Légalement, les belges sont égaux. S'il a une réglement qui crée une différence de traitement par cette situation identique -> Problème de légalité.\\

Donc, la commune peut taxer OUI. Mais si différence de traitement, ... cela doit être justifié.\\

\section{Tutelle}

 En RW, pour les réglements de taxe, il s'agit d'une tutelle d'approbation préalable. (Région Wallone ou Province). Tutelle d'annulation, pour les additionnels IPP, précompte immob. (entrée en vigueur article 190 de la Constitution). Un texte de loi voté mais pas publié ou mal publié (avant approbation) n'est pas en vigueur.\\

Quid de la publication des communes / Province (il y a le memorial administratif). Pour le moment, affichage dans un lieu accessible au public ... qui de la preuve ? Un AR prévoit que la mention de la publication doit être consigné dans un registre. (Registre des publications) Risque de faux en écriture publique. -> Copie du registre ou par toute voie de droit (pas d'attestation du Secrétaire communal).\\

Délai de 5 jours ouvrables avant d'entrée en vigueur. Possibilité de raccourcir le délai. Permis mais cela doit être spécifié dans le dispositif du réglement Taxe. \\

\begin{itemize}
\item Taxe directes (opération sur une période) -> Rétro-activité jusqu'au dernier jour de la période imposable pour l'entrée en vigueur (!= exercice d'imposition) cfr additionnels IPP
\item Impôts indirect (pèse sur un évènement, un fait générateur) -> Pas d'application du réglements pour la période rétro-active (Article 2 du Code Civil)
\end{itemize}

\null


Loi de 1841 : Obligation de mettre en place des voies de communication par les communes (construction des route au budget communal). Ouvertures des portes. Point de passage aux remparts reste un point de collecte (les octrois - taxe indirecte sur la circulation des marchandises à destination de clients).\\

Révolution. Interêt de la proximité des charbonnage et du chemin de fer. -> Inadéquation du système fiscal des octrois.Utilisation du pouvoir de limitation des communes puis aux provinces (selon la jurisprudence) Loi 18.07.1860 qui va interdire aux communes de prélever des octrois. -> Fin des ramparts car plus d'intérêts. Urbanisation devient grandissante. \\

Depuis 1860, 2 méchanismes :\\


\begin{itemize}
\item Utilisation du méchanisme de limitation de la fiscale locale
\item les communes gémissent -> Loi 1860 du fonds communal -> Fonds des communes. 
\end{itemize}
\null

Si limitation -> Octroi d'un complément au fonds\\

05.07.1860 -> Création du Crédit Communal. Emprunter avec un meilleur taux avec la garantie de l'Etat central\\

Utilité de connaitre la loi sur l'interdiction de l'octroi. Taxe indirecte sur la circulation de marchandise (Production de charbon, abattage sur le bois et vente).\\

Attention, le dépot de publicité n'est pas un octroi car gratuite, le dépôt est un service est par une circulation.\\

Aspiration de taxes (hold-up sur les recettes fiscales communales après la guerre) sur les véhicules à moteur. Mais suppléments de décimes additionnels (cfr les T.C.)\\




















\nocite{*}
\bibliographystyle{plain}
\bibliography{publications2}       % 'publications' is the name of a BibTeX file
\addcontentsline{toc}{chapter}{Bibliographie} 
\end{document}