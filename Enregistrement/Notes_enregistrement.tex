\documentclass{book}
\usepackage[utf8]{inputenc}
\usepackage[T1]{fontenc}
\usepackage{lmodern}
\usepackage[a4paper,left=3.5cm,right=2.5cm,top=2.5cm,bottom=2cm]{geometry}
\usepackage[frenchb]{babel}
\usepackage{fncychap}
\usepackage{fancyhdr}
\usepackage{sectsty}
\usepackage[pdftitle={Master complémentaire en droit fiscal - Droits d'enregistrement et de succession},pdfauthor={Laurent RICHARD}, pdfsubject={Notes de cours}, pdfkeywords={ULg, Fisc}]{hyperref}
\usepackage{graphicx}
\usepackage{pdfpages}
\usepackage{listings}
\usepackage{lscape}
\allsectionsfont{\sffamily}
\usepackage{fancybox}
\setcounter{secnumdepth}{4}
\usepackage{latexsym}
\usepackage{pifont}
\renewcommand\FrenchLabelItem{\textbullet}
\widowpenalty=9999
\clubpenalty=9999




\begin{document}
\sffamily
\newcommand{\RPoint}{\protect\includegraphics[height=1.7ex,keepaspectratio]{point.png}}
\newcommand{\RSave}{\protect\includegraphics[height=1.7ex,keepaspectratio]{Save.png}}
\renewcommand\labelitemii{\ding{220}}
\begin{titlepage}

\begin{center}
\begin{Large}ULg - Master complémentaire en droit fiscal\end{Large}\\
\vspace{0.5cm}- \\
\vspace{0.5cm}
Fiscalité des entreprises

\end{center}
\vspace{6cm}

\LARGE
\begin{center}
\textsc{Notes de cours - Droits d'enregistrement et de succession}\\
\end{center}

\vspace{10.0cm}

\normalsize
\flushright{
\textbf{Laurent RICHARD}\\
Comptable-fiscaliste agréé IPCF\\
\vspace{0.5cm}
Année académique 2012-2013}

\end{titlepage}

\newpage

\thispagestyle{empty}
\setcounter{page}{0}
\null

\newpage
\thispagestyle{empty}
\setcounter{page}{0}
\vspace{20cm}

\vfill
\begin{flushright}
Life is a succession of lessons, \\
which must be lived to be understood. \\ 
--- \textit{Ralph Waldo Emerson}
\end{flushright}
\vfill
\newpage

\renewcommand{\chaptermark}[1]{\markboth{#1}{}}
\renewcommand{\sectionmark}[1]{\markright{\thesection\ #1}}
\fancyhf{} \fancyhead[LE,RO]{\bfseries\thepage}
\fancyhead[LO]{\bfseries\rightmark}
\fancyhead[RE]{\bfseries\leftmark}
\renewcommand{\headrulewidth}{0.5pt}
\addtolength{\headheight}{0.5pt}
\renewcommand{\footrulewidth}{0pt}
\fancypagestyle{plain}{ \fancyhead{}
\renewcommand{\headrulewidth}{0pt}}

\thispagestyle{empty}
\setcounter{page}{0}
\null
\newpage
\pagenumbering{roman} \setcounter{page}{1} 




\tableofcontents



\chapter{Cours du 20.10.2012}
\pagenumbering{arabic} \setcounter{page}{1} 

\section{Introduction}

\subsection{Matrice cadastrale}

La matrice cadastrale est un registre qui, par commune ou division cadastre de commune, mentionne chaque propriétaire, les parcelles qu’il possède, les contenances et les revenus de ces parcelles.\\

Certaines informations sont d'ordre privées et d'autre publiques; c'est pourquoi nous parlerons donc d'extrait de la matrice cadastrale.\\
 
Quelles informations sont disponibles ?

\begin{itemize}
\item     Le nom et l'adresse du propriétaire
\item     Tous les autres biens qui lui appartiennent dans la même division cadastrale - situation au 1er janvier (demande écrite au bureau du cadastre de la direction régionale)
\item     Nature du bien (appartement, villa, bureau…)
\item     La section
\item     Le n° parcellaire
\item     L’année de construction
\item     La superficie de la parcelle
\item     Le revenu cadastral non indexé
\end{itemize}

\subsection{Fiche cadastrale}\\

\subsection{Revenu cadastral} 

(472 - 504 CIR)\\
  
Plus de péréquation depuis 1975. Revenus net qu'on peut retirer. Controle de la valeur vénale par les experts du Cadastre. (6 pour la province)\\

\subsection{Droit d'enregistement}

Droit d'enregistrement à l'accomplissement d'un acte. Droit occasionnel.\\

Sont appelés impots régionaux, les recettes fiscales expressément à l'article 1 & 2,2\degre de la loi spéciale du 16 janvier 1989 modifié par.

Les droits d'enregistement vont à la Région de meme que les hypotheque, ...

Région : Responsable taux, base imposable, exonération, ... 
Féderal : règle général de perception, les moyens de preuve, la matière imposable, procédure, droit de bail. 

L'essentiel des droits d'enregistrement sont versés aux Régions.

\subsection{Localisation}  

Circulaire 12/2005


Les droits d'enregistement et les droits de succession vont ensemble. Ils seront probablement repris pour 2015 par la Région flamande.\\

De plus en plus, une partie de la récupération de PrI flamand est récupérée par le privé.\\

DLU, les banques ont perçus l'impot. Quid de la reprise de la fonction de receveur par les notaires ? Quid du controle.  \\

Domaine : Secal, Ventes de biens saisis et de biens de l'Etat (hors immeuble), droit de chasse, taxe assurance, succession en déhérence, récupération de créance indue chomeur, ...\\

Bureau des actes judiciaire : exploits d'huissier (droit fixe de 25 EUR), jugement judiciaire, droit de condamnation de 3\%, ...\\

Notaire : Droit exigible immédiatement. -> Donation et testament applicable sans enregistrement.

Bureau des amendes pénales : Amendes et frais de justice, ... Taux de récupération faible.\\

Droit d'hypothèque 3 pour mille.\\

Taxe sur les ASBL 0,17\% sur le patrimoine de plus de 25.000 EUR.\\ 

Comité d'acquisition : vente de biens immeubles de l'Etat et des expropriation. Pas besoin d'acte notarié -> Gain d'argent pour l'acquéreur. Création en RW d'un nouvel organisme : Instruction 15 de 2009 (30.04.2009) Commission wallonne des transactions immobilières mais cela ne fonctionne pas.

\subsection{Transcription}

Déclaratif (partage d'immeuble) ou Acte translatif. 
Droit direct (droit direct sur la chose). Droit réel (Propriété, usufruit, ...). Droit personnel (le bailleur droit rendre le bien utilisable pour que le locataire puisse en jouir).

Tant qu'ils pas transcrit, ils ne sont pas opposables aux tiers. Les hypothèques n'accepteront que les actes authentiques (notaires, jugements,...)\\

Minute (La minute est l'original de l'acte notarié. Le notaire doit conserver la minute.) / Expédition (Une expédition est une copie certifiée conforme de l'acte original par le notaire. L'expédition est signée par le notaire et revêtue de son sceau.)\\

Grosse de l'acte notarié.(une seule grosse) - Dispense d'aller en justice. La grosse est une copie d'un acte authentique pourvue de la formule exécutoire. Ce caractère exécutoire donne la possibilité au créancier de prendre immédiatement des mesures empêchant que le débiteur ait encore l'occasion de se rendre insolvable. L'intervention d'un tribunal n'est plus nécessaire.\\

Inscription Hypothécaire (30 ans). Radiation possible. Demander un état hypothécaire. Le système est personnel et non réel (demande par rapport à la personne et non par rapport à la localisation du bien).\\

Caisse des dépots et consignation. Comptes dormants, succession en souffrance 

La mutation par décès n'est pas transcrite. -> Registre des propriétaires.

Service de recherche et de documentation - estimation des titres de sociétés non cotées.

Receveur - Délivre les attestations d'hérédité. (gratuit)

Faire les casus et l'envoyer au prof (eric.bruyere@skynet.be)

\chapter{eee}

Ecrit : instrumentum 
Operation juridique : negotium

Scanning des baux puis destruction.\\

La documentation immobilière est complète au contraire des hypothèques.\\

La documentation sur les biens meubles devient importante (donations, ...) -> Cadastre des fortunes mobilières.\\

Clause d'accroissement : La clause d’accroissement peut être analysée comme une convention par laquelle les acquéreurs d’un bien indivis cèdent réciproquement leur part (en pleine propriété ou en usufruit) à l’autre sous la condition suspensive de leur prédécès par rapport à l’autre.\\

Tontine : La clause de tontine est une stipulation contractuelle par laquelle les acquéreurs acquièrent du vendeur une part du bien sous condition résolutoire du prédécès de l’un par rapport à l’acquéreur de l’autre part, cette même part étant acquise sous condition suspensive de la survie de cet acquéreur par rapport à l’autre. \\


\nocite{*}
\bibliographystyle{plain}
\bibliography{publications2}       % 'publications' is the name of a BibTeX file
\addcontentsline{toc}{chapter}{Bibliographie} 
\end{document}