\documentclass{book}
\usepackage[utf8]{inputenc}
\usepackage[T1]{fontenc}
\usepackage{lmodern}
\usepackage[a4paper,left=3.5cm,right=2.5cm,top=2.5cm,bottom=2cm]{geometry}
\usepackage[frenchb]{babel}
\usepackage{fncychap}
\usepackage{fancyhdr}
\usepackage{sectsty}
\usepackage[pdftitle={Master complémentaire en droit fiscal - Principes généraux du Droit fiscal},pdfauthor={Laurent RICHARD}, pdfsubject={Notes de cours}, pdfkeywords={ULg, Fisc}]{hyperref}
\usepackage{graphicx}
\usepackage{pdfpages}
\usepackage{listings}
\usepackage{lscape}
\allsectionsfont{\sffamily}
\usepackage{fancybox}
\setcounter{secnumdepth}{4}
\usepackage{latexsym}
\usepackage{pifont}
\renewcommand\FrenchLabelItem{\textbullet}
\widowpenalty=9999
\clubpenalty=9999




\begin{document}
\sffamily
\newcommand{\RPoint}{\protect\includegraphics[height=1.7ex,keepaspectratio]{point.png}}
\newcommand{\RSave}{\protect\includegraphics[height=1.7ex,keepaspectratio]{Save.png}}
\renewcommand\labelitemii{\ding{220}}
\begin{titlepage}

\begin{center}
\begin{Large}ULg - Master complémentaire en droit fiscal\end{Large}\\
\vspace{0.5cm}- \\
\vspace{0.5cm}
Fiscalité des entreprises

\end{center}
\vspace{6cm}

\LARGE
\begin{center}
\textsc{Notes de cours - Principes généraux de Droit fiscal}\\
\end{center}

\vspace{10.0cm}

\normalsize
\flushright{
\textbf{Laurent RICHARD}\\
Comptable-fiscaliste agréé IPCF\\
\vspace{0.5cm}
Année académique 2012-2013}

\end{titlepage}

\newpage

\thispagestyle{empty}
\setcounter{page}{0}
\null

\newpage
\thispagestyle{empty}
\setcounter{page}{0}
\vspace{20cm}

\vfill
\begin{flushright}
Life is a succession of lessons, \\
which must be lived to be understood. \\ 
--- \textit{Ralph Waldo Emerson}
\end{flushright}
\vfill
\newpage

\renewcommand{\chaptermark}[1]{\markboth{#1}{}}
\renewcommand{\sectionmark}[1]{\markright{\thesection\ #1}}
\fancyhf{} \fancyhead[LE,RO]{\bfseries\thepage}
\fancyhead[LO]{\bfseries\rightmark}
\fancyhead[RE]{\bfseries\leftmark}
\renewcommand{\headrulewidth}{0.5pt}
\addtolength{\headheight}{0.5pt}
\renewcommand{\footrulewidth}{0pt}
\fancypagestyle{plain}{ \fancyhead{}
\renewcommand{\headrulewidth}{0pt}}

\thispagestyle{empty}
\setcounter{page}{0}
\null
\newpage
\pagenumbering{roman} \setcounter{page}{1} 




\tableofcontents



\chapter{Cours du 22.09.2012}
\pagenumbering{arabic} \setcounter{page}{1} 

\section{Introduction}

Examen oral - 2 questions (Généralement au mois de décembre).

Partie VII -> Pipeline

\chapter{Définitions, techniques et notions générales}


Arrêt Cours de Cassation (30.11.1950)\\

\textit{L'impôt constitue un prélèvement pratiqué par voir d'autorité par l'Etat, les provinces ou les communes sur les ressources des personnes vivant sur leur territoire ou y possédant des intérêts, pour être affecté aux services d'utilité générale}\\

\begin{itemize}
\item IPP, on présume la résidence via le Registre d'état civil -> Taxation
\item A priori, l'impôt collecté alimente les caisses de l'Etat sans attribution spécifique.
\item Ressources != revenus
\item Personne physique et morale
\end{itemize}
\null
Quid des régions et des communautés mais c'est normal vu la date de l'Arrêt. Le prélèvement se fait en argent. (pas en nature).\\

Rapport avec

=> Redevance (rémunération/contre-valeur d'une prestation qui est personnelle, qui est individualisée et qui est demandé par le contribuable). Une redevance n'est pas un impôt (Tribunal de 1ere instance) mais dépend selon le montant du Tribunal de Justice de Paix. Les principes consitutionnels ne s'appliquent pas aux redevances vu que ce n'est pas un impôt.

=> Cotisation parafiscale (ONSS patronale, ...) Proche de l'impôt vu que ce n'est pas solicité mais affecté directement à un budget précis (ici de la sécurité sociale).

\section{Différents types d'impôts}


\begin{itemize}
\item En fonction de l'objet : sur le revenu, sur la fortune, sur la dépense ou sur la consommation; (ISF en France : 1,3 Millions de patrimoine)
\item Directs ou indirects : selon la Cour de Cassation \textit{l'impôt direct frappe une situation durable par nature, présnetant une certaine stabilité, tandis que l'impôt indirect frappe une situation ou une opération qui n'est que passagère}
\item En fonction de la façon dont il est calculé

\end{itemize}

\section{L'assiette de l'impôt et les concepts qui en découlent}

\begin{itemize}
\item L'assiette (ou la matière imposable) : fondement économique de l'impôt, sa cause. (Transmission du patrimoine, passage de frontière, contrat de vente relatif à un immeuble, ...) -> Concept théorique différent de la base.
\item Base : montant qui va etre frappé d'un taux par le législateur.
\item exemption, exonération, immunisation. (exclusion dès le départ de la base imposable) -> Différent de la déduction ou de l'abattement. (exlusion de la base imposable d'un montant qui avait été préalablement inclus.
\end{itemize}

Convention Préventive de Double Imposition : Traité permettant de se répartir le pouvoir d'imposition. Exonération du montant succeptible d'etre imposé des deux cotés. En Belgique, c'est une exonération sous réserve de progressivité. La cour de Cassation a accepté le principe.

\section{Le fait générateur et l'exigibilité de l'impôt}

Le fait générateur, est l'evenement ou l'ensemble de circonstances qui font naitre la dette d'impôt. Il peut etre factuel, une evenement juridique (contrat).\\

Exigibilité : Moment auquel l'impôt doit etre payé. Exemple éclairant de l'article 17 §1er du C.T.V.A. : " Pour les livraisons de biens, le fait générateur de la taxe intervient et la taxe devient exigible au moment où s'opère la livraison du bien. (Principe) \\

Toutefois, lorsque le prix est facturé ou encaissé, en tout ou en partie, avant ce moment, la taxe devient exigible, selon le cas, au moment de la délivrance de la facture ou au moment de l'encaissement, sur la base du montant facturé ou encaissé. (Exception)\\

=> Délai de payement (Exigible != payable tout de suite) Il peut etre légal ou arrangement avec le receveur.

\section{Le redevable, sensu lato, de l'impôt}

Personne qui doit payer à l'autorité taxatrice.\\

=> Différence fondamentale entre l'obligation et la contribution à l'impôt.\\

Le redevable doit l'impôt, obligé à l'impôt. Il sera saisi par le receveur si l'impôt n'est pas payé en temps et en heure. L'obligation à l'impôt est généralement imposé par la Loi et on ne peut y déroger.\\

La contribution à la dette est la question de savoir entre les parties qui sont intéressée par l'opération se répartissent entre elles l'impôt.  (donc nécessité d'avoir plusieurs parties). Généralement laissé à l'appréciation des parties.\\

Vente d'un immeuble, la contribution à la dette est relativelent libre. En pratique, l'acheteur paye les frais (Code civil) mais les parties peuvent convenir d'autre chose. Dans certains cas, la contribution n'est pas libre (par la Loi) cfr le précompte immobilier dans le cadre d'un bail de résidence principale.\\

Les précomptes mobilier et professionnel. Le redevable est le débiteur du revenu. (L'employeur pour le PrP et l'emprunteur/société pour le PrM) Obligation à la dette. Concernant la contribution, il n'y a pas vraiment de question.\\

Concernant la vente d'un bien, fiscalement, le compromis vaut vente. Il est juste donné un délai pour l'acte notarié et payement du droit d'enregistrement.

\section{La fixation de l'impôt du}

\begin{itemize}
\item impôt fixe - Montant absolu. (Taxe sur les serveuses à Liège - 2500 EUR par an et par serveuse) Non lié au chiffre d'affaires. On ne tient pas compte de la capacité contributive.
\item impôt ad valorem - Montant est le résultat de l'application d'un taux/pourcentage à une valeur déterminée qui est la base imposable. 
\end{itemize}
\null
La base d'imposition\\

\begin{itemize}
\item Taux Proportionnel - un seul taux
\item Taux progesstion
\begin{itemize}
\item Progressif globalement (un seul taux celui de la tranche dans laquel on se trouve globalement) 
\item Progressif par tranche (le taux varie selon la tranche de revenu spécifique).
\end{itemize}
\item Le taux marginal est le taux subit dans un bareme par tranche et qui est le taux le plus élevé.
\item Le taux moyen est le rapport entre l'impôt payé effectivement et la base d'imposition total.
\item Additionnels/soustractionnel : impôt s'ajoutant/se soustrayant calculé sur un autre impôt.
\item Crédit d'impôt est une somme à valoir, un montant que le débiteur de l'impôt peut imputé sur l'impôt qu'il doit (impôt payé à l'étranger sur ce revenu).
\item L'impôt négatif serait l'hypothèse où en bout de course, ce n'est va vous qui devez de l'argent mais que cela soit l'Etat qui vous devrait de l'argent.
\end{itemize}


\section{la naissance de l'impôt et ses conséquences}

Le rôle - la liste exhaustive de toutes les personnes qui sont résidents fiscaux belges. (Nom, prénom, adresse, Numéro national).\\

L'enrolement est le fait de porter dans le role le montant d'impôt qu'une personne doit. (impôt direct mais pas indirect). Sauf si les indirects sauf s'ils ne sont pas payé et donc à mettre en recouvrement.\\

L'enrolement fait naitre la dette d'impôt. Mais cela ne rend pas le role executoire. Il le sera quand lorsqu'un fonctionnaire d'un grade suffisant aura porter son visa, rendu executoire. Le fait que que l'enrolement soit rendu executoire est le début pour le prescription. \\

Le privilege du préalable signifie que l'Administration fiscale peut executer au sens technique du terme le débiteur de l impôt non payé en d'autre termes, peut pratique une saisie à charge de ce débiteur ne payant pas l'impôt du. L'administration n'a pas besoin de titre exécutoire pour effectuer la saisie, pas besoin de jugement. (au contraire du droit commun - Exception : l'acte notarié est un acte authentique exécutoire.)\\

Le cas des précomptes. Ils ne sont pas enrolés. Ils ne sont enrolés que s'ils ne sont pas payés. Concernant le précompte immobilier est enrolé.\\

\chapter{Les sources du droit fiscal belge}

\section{Les sources formelles}

Hiérarchie des normes. -> Pyramide des normes.\\

Toutes les normes inférieures doivent respecter les normes supérieures.\\

Droit International (conventions fiscale ou non, bilatéral ou multi-latéral) - C.E.D.H

-> 

Droit Européen - T.F.U.E., réglements spécifiques

->

Constitution belge

-> 

Loi (et Décret) \& P.G. du droit, principalement les P.G. de Bonne Administration.

->

Mesures d'exécution (Arrêtés royaux et ministériel)

->

Réglements provinciaux et communaux

->

Circulaires (Si elle explique la Loi, OK. Si elle étend la Loi, la circulaire est illégale)\\


Quid collaboration du contribuable VS Article 6 de CEDH - non incrimination

\section{Autres sources}

Jurisprudence

->

Question parlementaire

-> 

Doctrine (ensemble des personnes qui ont +/- d'influence dans le domaine)

\chapter{Principes constitutionnels}

\section{Légalité de l'impôt}

Article 170 de la Constitution\\

Art. 170. § 1. Aucun impôt au profit de l'Etat ne peut être établi que \textbf{par} une loi.\\

  § 2. Aucun impôt au profit de la communauté ou de la région ne peut être établi que \textbf{par} un décret ou une règle visée à l'article 134. La loi détermine, relativement aux impositions visées à l'alinéa 1er, les exceptions dont la nécessité est démontrée.\\

  § 3. Aucune charge, aucune imposition ne peut être établie par la province que \textbf{par} une décision de son conseil. La loi détermine, relativement aux impositions visées à l'alinéa 1er, les exceptions dont la nécessité est démontrée. La loi peut supprimer en tout ou en partie les impositions visées à l'alinéa 1er.\\

  § 4. Aucune charge, aucune imposition ne peut être établie par l'agglomération, par la fédération de communes et par la commune que \textbf{par} une décision de leur conseil.  La loi détermine, relativement aux impositions visées à l'alinéa 1er, les exceptions dont la nécessité est démontrée.\\
  
  Il faut toujours une décision des représentants des contribuables potentiels.\\
  
  En vertu d'une loi (la loi donne le principe). Par une loi (la loi votée doit contenir le principe de l'impôt mais aussi ses éléments fondamentaux).\\
  
  L'AR 20 TVA a été avalisé par le Parlement. En principe, on délègue pas les principes fondamentaux.\\
 
 On ne peut pas faire d'accord sur des questions de droit. Uniquement sur des questions de fait.\\
 
Art. 172. Il ne peut être établi de privilège en matière d'impôts. Nulle exemption ou modération d'impôt ne peut être établie que par une loi. \\


\section{Annualité de l'impôt}

  Art. 171. Les impôts au profit de l'Etat, de la communauté et de la région sont votés annuellement.
  Les règles qui les établissent n'ont force que pour un an si elles ne sont pas renouvelées.

\section{Egalité devant l'impôt}

Deux situations identiques doivent générer un meme impôt. Deux situations différentes doivent générer un impôt différent.


\chapter{Cours du 29.09.2012}

\section{Autonomie relative du droit fiscal}

L'autonomie du droit fiscal n'existe que si le législateur en décide ainsi. Autonomie -> Ne renverrait pas au droit commun.\\

Article 471 du CIR 92\\
   § 1er. Il est établi un revenu cadastral pour tous les biens immobiliers bâtis ou non bâtis, ainsi que pour le matériel et l'outillage présentant le caractère d'immeuble par nature ou d'immeuble par destination.\\

   § 3. Par matériel et outillage, on entend, à l'exclusion des locaux, abris et des accessoires indispensables à ces derniers, tous appareils,machines et autres installations utiles à une exploitation industrielle,commerciale ou artisanale.\\


Les biens sont immeubles, ou par leur nature, ou par leur destination, ou par l'objet auquel ils s'appliquent (Art. 517 du Code civil). Sont immeubles par nature, les fonds de terre et les bâtiments (art. 518 Code civil).\\

Aux bâtiments, doivent être assimilés les objets qui s'y unissent ou s'y incorporent d'une manière durable et habituelle. Les critères retenus sont l'intensité de l'incorporation du bien ainsi que son caractère durable.\\

La Cour de cassation a toutefois rejeté la thèse de l'incorporation indissociable pour établir un bien immeuble par incorporation. Un bien peut donc être immeuble par nature, même s'il peut être enlevé sans détérioration, mais il doit présenter un caractère durable et habituel (Cfr. banques de jurisprudence).\\

Exemple (Article 9 du CTVA) : Les titres au porteur ne sont pas considérés comme des biens corporels.\\

\section{Règles d'interprétation généralement utilisées}

Attention, il s'agit d'une cascade de règles :\\

\begin{itemize}
\item Un texte clair ne s'interprète pas;
\item On donne aux mots leur sens usuel, s'il n'y a pas de définition spécifique qui soit donnée par le texte lui-même, ou le droit fiscal en génétral (autre droit) ou encore le droit commun (droit civil ou droit privé)
\item recours éventuel à l'intention du législateur (les «~travaux préparatoires~») - les actes qui ont aidé à la discussion, les discussions avant le vote.
\item Pas de recours possible à l'interprétation par analogie ! (par comparaison)
\item L'interprétation, quand elle a lieu, doit se faire de manière structe (pas de portée trop loareg ni trop restreinte.)
\item En cas de doute, application de l'adage «~in dubio, contra fiscum~»
\end{itemize}

\subsection{Un texte clair ne s'interpréte pas}

Exemple : le recours à la procédure d'imposition d'office en matière de taxes locales :\\

Article 6 Loi du 24.12.1996 sue l'établiessement et le recouvrement des taxes locales : (idem article L3321-6 Code wallon de la démocratie locale et de la décentralisation)\\

\textit{Lorsque le règlement de taxation prévoit une obligation de déclaration, la non-déclaration dans
les délais prévus par ce même règlement ou la déclaration incorrecte, incomplète ou imprécise de la part du
redevable \textbf{entraîne} l’enrôlement d’office de la taxe.}

\null

A comparer avec 351 CIR\\

\textit{L' administration \textbf{peut} procéder à la taxation d'office en raison du montant des revenus imposables qu'elle peut présumer eu égard aux élémentsdont elle dispose, dans les cas où le contribuable s'est abstenu :}
   
   \subsection{Pas d'analogie}
   
   Cfr Arrêt C. Appel de Liège 25.10.2000. Pas d'analogie de la règle des 5000 km ATN aux indépendants (règle valable pour les employés et dirigeants d'entreprise)
   
      \subsection{L'interprétation, qaund elle a lieu, doit se faire de manière stricte}

Article 418, 419 (remboursement d'impôt)\\

Anatocisme (1154 du C. Civil) - Une mise en demeure manifestant la volonté de mettre des intérets sur les intérêts peut activer.\\

En droit fiscal, pas d'anatocisme.\\

Arrêt C. Appel Mons, 16.03.2005 confirmé C. Cassation 18.06.2010 \& 24.05.2012 (cfr juridat.be)

\subsection{In dubio, contra fiscum}    
   
   Article 64
   
   Une exception doit toujours être interprétée de manière RESTRICTIVE. 
   
\chapter{Evitement de l'impôt, et les moyens de le combattre pour l'Administration}
   
\begin{itemize}
\item Les moyens d'éviter l'impôt
\item La fraude fiscale et l'évasion fiscale
\item Les moyens d'action de l'administration en cas d'évitement licité d'impôt - Necessité d'une intervention législative
\end{itemize}


\section{Les moyens d'éviter l'impôt} 
   
Listing \underline{sans} jugement de valeur :\\
   
\begin{itemize}
\item l'abstention;
\item la dissimulation (en tout ou en partie);
\item la simulation;
\item la substitution;
\item l'accomplissement d'opérations non simulées, mais à des conditions anormales. (Prix de transfert)
\end{itemize}
 \null  
   Toute simulation implique une dissimulation. Le contraire n'est pas vrai. La Cour de Cassation a construit pas mal de jurisprudence.\\
   
   Il y a simulation lorsque par un acte secret(= «~la contre-lettre~»), les parties à un acte juridique (= «~l'acte apparent~») décident entre elles de détruire ou de modifier les effets de cet acte, qui n'existe que pour être opposé aux tiers.\\
   
   La simulation peut porter sur les modalités de l'acte (partie en noir), Quant à sa nature (Donation contre vente) ou sur l'existence même de cet acte (fausse facture)\\
   
   Cfr Coup de l'accordéon (Remboursement du capital puis incorporation de réserves au Capital).\\
   
   Prix de transfert (coût anormal pour augmenter les charges en Belgique et augmenter le bénéfice dans un autre pays ou paradis fiscal).\\


\section{Différence entre fraude fiscale et évasion fiscale}

Si l'accordéon est une recherche d'un diminution ou une absence de taxation, il n'y a pas de fraude mais une évasion. Si les moyens sont illégaux, c'est une fraude.\\

=> Licéité du procédé fait la différence entre fraude et evasion. (Après c'est une question morale.)\\

Les conséquences en cas de fraude (régularisation sanctions administratives / fiscales et potentiellement pénales - Juge d'instruction et suite voulue en corerctionnelle) Cfr 200\% d'amende TVA. \\

\section{Quels sont les critères de la licéité ?}

\begin{itemize}
\item L'abstention n'est pas répréhensible;
\item par contre, l'altération volontaire de la réalité est condamnable;
\item le choix de la voie la moins imposée - la jurisprudence de la Cour de cassation (Brepols)
\end{itemize}

Fraude à la loi

Le fait de se placer dans le champs d'action d'une loi qui n'est normalement pas applicable dans le but unique d'être dans son champ et d'en bénéficier.\\

Ex : Delaware (responsabilité limitée et capital minimum)

=> On refuse les effets de la loi et on applique le régime normal si la loi n'était pas appliqué.\\

Quid absorption dans le cas des pertes fiscales. En cas de dispariation de A (avec des pertes), disparition des pertes. (INES) -> La société malsaine absorbe la société saine. Ce qui est illogique économiquement.\\

La théorie de la fraude à la loi est écartée du droit fiscal.\\

Cour de Cassation – Arrêt du 6 juin 1961 - Brepols\\

\textit{Attendu, d’autre part, qu’il n’y a ni simulation prohibée à l’égard du fisc, ni partant fraude fiscale, lorsque, en vue de bénéficier d’un régime, fiscal plus favorable, les parties, usant de la liberté des conventions, sans toutefois violer aucune obligation légale, \textbf{établissent des actes dont elles acceptent toutes les conséquences}, même si la forme qu’elles leur donnent n’est pas la plus normale;}\\

Arrêt - Au Vieux Saint Martin - Cass., 22/03/1990\\

\textit{Il n'y a ni simulation prohibée à l'égard du fisc, ni partant fraude fiscale, lorsque (…) les parties (…) établissent des actes dont elles acceptent toutes les conséquences,\textbf{ même si ces actes sont accomplis à la seule fin de réduire la charge fiscale}}\\

Note : il y a toute fois mention des mots (sans toutefois violer aucune obligation légale). Quid si erreur comptable -> erreur/obligation légale. Quid si mauvaise comptabilisation ? Quid si comptable extérieur ?\\

=> Respecter toutes les obligations.\\


Les théories des réalités économiques : condamnation par la Cour de Cassation.


\section{Les droits du fisc en cas de simulation.}

\subsection{C.Civil 1321}
C. Civil 1321 - Les contre-lettres ne peuvent avoir leur effet qu'entre les parties contractantes : elles n'ont point d'effet contre les tiers. (Non applicable en droit fiscal)\\

Cass 04.01.1991 - Les impôts doivent etre établies sur la base de la construction juridique réellement utilisée.\\

Dérogations :\\

\begin{itemize}
\item Droit d'enregistrement - Taxation sur ce qui est présenté.
\item Droit de succession - Taxation ne peut etre diminuée par la contre-lettre. (Art 106)
\end{itemize} 

\subsection{L'action en déclaration de simulation : dérogation fiscale au droit commun}

Devant le Tribunal de 1ère Instance. - Pas d'obligation pour le fisc. Il doit invoquer la simulation et la prouvé mais pas d'action en justice nécessaire.

\section{Les moyens d'action de l'Administration en cas d'évitement licite d'impôt - Nécessité d'une intervention législative}

Si pas de simulation, la seule méthode de rectifictaion est de se baser sur un texte légal. -> Intervention du législateur. Trouver un texte suffisamment général et bien rédigé ou un texte d'intervention ponctuelle.

\subsection{Mesures générales - Requalification ou inopposabilités de qualification}

Article 344 ancien - Evasion fiscale et non Fraude fiscale.\\

Notion de qualification juridique. Pour une même opération, il faut deux qualifications possibles (et similaires). Beaucoup de problèmes, car point de vue juridique, c'est rare voir rarissime. (même vente/donation n'est pas applicable).\\

Nouvel 344 §1er CIR 92. Non opposabilité de l'acte juridique. Vise l'évasion fiscale ! Pas de sanction fiscale autre que le redressement.\\

Cfr Art 1, §10 TVA ert Artcile 79 §2 CTVA.\\

\chapter{Cours du 06.10.2012}

Essentiel de savoir qui dit quoi ... Important que connaitre la compétence des différents tribunaux et entre autre le tribunal de 1ere instance.\\

Avant, directement en C. Appel (point de vue CD) suite à un recours contre la décision du Directeur régional.  Donc un seul degré et pas d'appel judiciaire possible. TVA, c'est l'opposition a contrainte devant le Juge de 1ere instance.\\

Depuis, alignement des CD sur le schéma "classique", à savoir, appel devant le Juge de 1ere instance.

C. Cassation vérifie la légalité de la décision de la Cour d'Appel. Matière à casser une décision qui n'aurait pas respecter les règles de procedure ou violer des principes légaux. Elle n'est donc pas un 3e degré de juridiction. Elle ne juge pas les faits mais le droit.\\

Acte individuel : Acte isolé (nomination, ...)\\

2eme compétence du Conseil d'Etat : Section législation. Donne des avis au législateur lors de la préparation d'une Loi. Avis du Conseil d'Etat. Obligatoire. Dispense en cas d'urgence (prétendue).\\

Attention à la différence des normes contrôlées par la C. Constitutionnelle et par le Conseil d'Etat.\\

La cour Const. (avant Cour d'Arbitrage) était là pour arbitrer entre les Région/Communautés et l'Etat fédéral. Puis ajout de compétences d'autres disposition, anciens 10, 11 et 17 (Egalité et liberté d'enseignement). Extention pour arriver au schéma actuel (8 à 32, 170, 172 et 192).\\

Cour consti (Loi et assimilés) - Conseil d'Etat (Actes administratifs en dessous de la Loi)\\

Les institutions judiciaire ne peuvent juger/vérifier de la constitutionalité d'une loi (Pas de controle du pouvoir législatif). Mais bien la confirmité d'un loi avec la législation européenne ou d'un acte administratif par arpport à la Loi ! => Cour constitutionnel n'est pas le pouvoir judiciaire -> Statut à part.\\

Jugement avec question préjudicielle. -> Question à la Cour -> Arrêt de la Cour constitutionnelle. Décision de violation -> Retour au Tribunal -> Jugement ne pouvant faire application de la norme légale contestée. Mais la norme reste applicable pour les autres personnes ne faisant pas partie de l'action en justice. => Contestation sur base de l'Arret du Conseil d'Etat. \\

Première Instance : Compétence résiduelle - Non attribué aux autres.\\

Jurisprudence en matière d'égalité avant 1989 (Création de la Cour constitutionnelle) : Cassation et Conseil d'Etat. Surtout point de vue communal et provincial . Cassation suite à un recours contre enrolement personnel. Conseil d'Etat pour suspension et/ou annulation "collective" (erga omnes).\\

Pour etre égalitaire, il faut un critère objectif et une vocation générale. (Cass 20.01.1958 - Pied-de-Boeuf)\\

Conseil d'Etat (09.06.1964), "doit avoir un rapport avec le but et la nature de l'impot." -> But : augmenter les recettes. Mais aussi le but ne justifie pas la différenciation entre les contribuable. Cfr justification de la taxe dans le texte de celle-ci. Des justifications auraient pu etre trouvées mais ne se trouvaient pas dans le texte.\\

Cass. 20.11.1975, taxation des parcelles non batie pour éviter la spéculation.  Exemption des sociétés publics de logements. La Cour se rallie au Conseil d'Etat. Contribuables différents -> Taxation différente. Pourvoi des sociétés privées rejetées.\\

CA, 21 mars 1995 et CA 18 février 1998\\
\begin{itemize}
\item Différence de traitement entre des catégories de personnes doivent reposer sur un critère objectif et raisonablement justifié.
\item Justification du but et des effets ainsi que la natures des principes.
\item Distinction ou son absence doit être raisonnablement justifiée, ce qui suppose un rapport de pertinence entre les deux termes posés ainsi qu'un rapport de proportionnalité.
\end{itemize}
\null

Critère objectif : le critère qui requiert constatation et non appréciation. (Précis et éviter de se fonder sur des notions vagues qui par leur nature impliquerait une appréciation nécessairement subjective)\\

Le critère, et donc la catégorie, doit etre pertinent. Si non pertinent -> On ne peut justifier la différence de traitement -> Viole la Constitution.\\

Cas des sociétés dormantes.\\

Cas de la notion de PME pour l'application du taux réduit à l'ISOC (article 215) - La Cour a annulé la disposition de la Loi de 2002 mais la maintient pour l'exercice 2004 car cela pourrait porter préjudice aux sociétés s'étant adapté sur base de la loi. Loi de réparation. Annulation non rétro-active.\\

La comparabilité des catégories : la C.A. a précisé que les catégories doivent etre comparables. IPP n'est pas comparable à l'ISOC. Il n'est donc pas anormal de différencier des personnes physiques et des personnes morales.\\

Motivation budgétaire de certaines lois peut-elle constitué une justification ? Lorsque des avantages sont octroyés à des contribuables, il ne suffit pas d'invoquer des impératifs budgétaires pour faire endosser par une autre catégorie de contribuables la neutralité budgétaire.\\

Ex : Décumul des revenus professionnels des epoux. Application immédiate pour les couples à 2 revenus et application à retardement pour les couples à 1 seul revenu. Les couples à 2 revenus allaient en bénéficier via une adaptation du bareme de precompte professionnel. Par contre les couples à un revenu ne serait pas adapter. Arreté Royal -> Conseil d'Etat. Discrimination.\\

EX 2 : Decret RW 22.10.2003 modifiant les droits de succession. Réduire les taux de droit de succession pour les personnes proches. Adaptation en augmentant le taux pour les autres. 90\% sur la tranche la plus élevée (autres personnes). Recours devant la Cour Constitutionnelle (22.06.2005). Aucune autre justification que budgétaire n'a  été invoquée que la neutralité budgétaire. La Cour rappelle que si on souhaite encourager/décourager, rien n'empeche de taxer mais ici, une augmentation pour des modif budgétaire n'est en rien pour dissuader des comportement. -> Problème d'égalité. Atteinte disproportionné aux droit du testateur et aux espérances légitimes qu'à le légataire de les recueillir.\\

















   
\nocite{*}
\bibliographystyle{plain}
\bibliography{publications2}       % 'publications' is the name of a BibTeX file
\addcontentsline{toc}{chapter}{Bibliographie} 
\end{document}